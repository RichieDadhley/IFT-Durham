\documentclass[11pt,oneside]{book}
\usepackage[margin=1.2in]{geometry}
\usepackage[toc,page]{appendix}
\usepackage{graphicx}
\usepackage{natbib}
\usepackage{lipsum}
\usepackage{caption}
\usepackage[T1]{fontenc}
\usepackage{titlesec, blindtext, color}
\usepackage{xcolor,tikz}
\usetikzlibrary{patterns}
\usetikzlibrary{decorations.markings}
\usetikzlibrary{decorations.pathmorphing}
\usepackage{amsmath,amssymb,amsthm,mathrsfs,amsfonts,xfrac,pifont,bbold,physics,enumitem}
\usepackage[utf8]{inputenc}
\usepackage{amsthm}
\usepackage[breakable, theorems, skins]{tcolorbox}
\usepackage[colorlinks = true,
            linkcolor = red,
            urlcolor  = blue,
            citecolor = red,
            anchorcolor = red]{hyperref}
\usepackage{cleveref}

\usepackage{simplewick}

\usepackage{soul}
\usepackage{frcursive}
\usepackage{booktabs}


\usepackage{array}
\newcolumntype{C}[1]{>{\centering\arraybackslash}m{#1}}


% -------------------------------------------------------------------
% Theorem Styles
% -------------------------------------------------------------------

\theoremstyle{definition} % Define theorem styles here based on the definition style (used for definitions and examples)
\newtheorem*{definition}{Definition}

\theoremstyle{plain} % Define theorem styles here based on the plain style (used for theorems, lemmas, propositions)
\newtheorem{theorem}{Theorem}[section]
\newtheorem{axiom}{Axiom}
\newtheorem{corollary}[theorem]{Corollary}
\newtheorem{lemma}[theorem]{Lemma}
\newtheorem{proposition}[theorem]{Proposition}

\theoremstyle{remark} % Define theorem styles here based on the remark style (used for remarks and notes)
\newtheorem*{notation}{Notation}
\newtheorem*{solution}{Solution}

\newtheoremstyle{underline}% name
{}        % Space above, empty = `usual value'
{}              % Space below
{}              % Body font
{}    % Indent amount (empty = no indent, \parindent = para indent)
{}              % Thm head font
{.}             % Punctuation after thm head
{1.5mm}         % Space after thm head: \newline = linebreak
{{\underline{\textit{\thmname{#1}\thmnumber{ #2}}~\thmnote{(#3)}\unskip}}}  % Thm head spec

\theoremstyle{underline}

\newtheorem{remark}[theorem]{Remark}
\newtheorem{example}[theorem]{Example}
\newtheorem{claim}[theorem]{Claim}
% -------------------------------------------------------------------
% Chapter Headings
% -------------------------------------------------------------------

\setcounter{chapter}{-1}

\makeatletter
\renewcommand{\@chapapp}{Lecture}
\makeatother
\definecolor{lightergray}{rgb}{0.9,0.9,0.9}

\usepackage{titlesec}
\titleformat{\section}{\large\bfseries\raggedright}{}{0em}{\colorsection}[\titlerule]
\titleformat{name=\section,numberless}{\large\scshape\bfseries\raggedright}{}{0em}{\colorsectionnonumber}[\titlerule]

\titleformat{\subsection}{\bfseries\raggedright}{}{0em}{\colorsubsection}
\titleformat{name=\subsection,numberless}{\bfseries\raggedright}{}{0em}{\colorsubsectionnonumber}

\newcommand{\colorsection}[1]{%
    \colorbox{lightergray}{\parbox{\dimexpr\textwidth-2\fboxsep}{\thesection\ \ #1}}}
\newcommand{\colorsectionnonumber}[1]{%
    \colorbox{lightergray}{\parbox{\dimexpr\textwidth-2\fboxsep}{#1}}}
    
\newcommand{\colorsubsection}[1]{%
    \colorbox{lightergray}{\parbox{\dimexpr\textwidth-2\fboxsep}{\thesubsection\ #1}}}
\newcommand{\colorsubsectionnonumber}[1]{%
    \colorbox{lightergray}{\parbox{\dimexpr\textwidth-2\fboxsep}{#1}}}
    
\definecolor{gray75}{gray}{0.75}
\newcommand{\hsp}{\hspace{20pt}}
\titleformat{\chapter}[hang]{\Huge\bfseries}{\thechapter\hsp\textcolor{gray75}{|}\hsp}{0pt}{\Huge\bfseries}

\input{Shortcuts.tex}

\begin{document}

\captionsetup[figure]{margin=1.5cm,font=small,labelfont={bf},name={Figure},labelsep=colon,textfont={it}}
\captionsetup[table]{margin=1.5cm,font=small,labelfont={bf},name={Table},labelsep=colon,textfont={it}}
\setlipsumdefault{1}

\frontmatter

\begin{titlepage}
	\centering
	    \scshape % Use small caps for all text on the title page
        \vspace*{\baselineskip} % White space at the top of the page
        
	    \rule{\textwidth}{1.6pt}\vspace*{-\baselineskip}\vspace*{2pt} % Thick horizontal rule
	    \rule{\textwidth}{0.4pt} % Thin horizontal rule
	    
	    \vspace{0.75\baselineskip} % Whitespace above the title
	    
	    {\LARGE Introduction To Field Theory} % Title
	    
	    \vspace{0.75\baselineskip} % Whitespace below the title
	    
	    \rule{\textwidth}{0.4pt}\vspace*{-\baselineskip}\vspace{3.2pt} % Thin horizontal rule
	    \rule{\textwidth}{1.6pt} % Thick horizontal rule
	    
        \vspace{5\baselineskip} % Whitespace after the title block

	    Course delivered in 2019 by 
	
	    \vspace{0.5\baselineskip} % Whitespace before the editors
	
	    {\scshape\Large Professor Michael Spannowsky} % Lecturer Name
	
	    \vspace{0.5\baselineskip} % Whitespace below the editor list
	
	    \textit{Durham University} % Lecturer Institution
	    
	    \vspace{5\baselineskip} % Whitespace after the title block
	    
	    \includegraphics[width=8cm]{images/DurhamLogo.png}\\[1cm] % Logo 
	    
	    \vspace{3\baselineskip}

	    Notes taken by 
	
	    \vspace{0.5\baselineskip} % Whitespace before my name
	
	    {\scshape\Large Richie Dadhley} % Lecturer Name
	   
	    \vspace{0.5\baselineskip} % Whitespace below my name
	    \textit{richie@dadhley.com} % Email
	
	    \vfill % Whitespace between editor names and publisher logo
\end{titlepage}



% -------------------------------------------------------------------
% Acknowledgements
% -------------------------------------------------------------------

\newpage
\section*{Acknowledgements}

These are my notes on the 2019 lecture course "Introduction to Field Theory" taught by Professor Michael Spannowsky at Durham University as part of the Particles, Strings and Cosmology Msc. For reference, the course lasted 4 weeks and was lectured over 24 hours. \\

I have tried to correct any typos and/or mistakes I think I have noticed over the course. I have also tried to include additional information that I think supports the taught material well, which sometimes has resulted in modifying the order the material was taught. Obviously, any mistakes made because of either of these points are entirely mine and should not reflect on the taught material in any way. \\

% For any other notable sources used:
I have also used/borrowed from Professor David Tong's notes throughout the course (particularly at the start). This notes are a must for any new students to QFT and can be found via 

\begin{center}
    \href{http://www.damtp.cam.ac.uk/user/tong/qft/qft.pdf}{http://www.damtp.cam.ac.uk/user/tong/qft/qft.pdf}
\end{center}

The material is presented in the order taught. This generally coincides with Professor Tong's notes, however there are small deviations here and there. Probably the biggest difference is the fact that we present the LSZ theorem in this course, whereas that is saved for the Advanced QFT course at Cambridge. Equally, normal ordering is not introduced until lecture 7, where Wick's theorem is discussed, whereas Prof. Tong introduces it much earlier (in order to drop the infinities in expectation values). 

I would like to extend a message of thanks to Professor Spannowsky for teaching this course. I would also like to thank Professor Tong for uploading his notes to the internet. \\

If you have any comments and/or questions please feel free to contact me via the email provided on the title page. \\

These notes are currently a work in progress, so for updated notes, as well as a list of other notes/works I have available, visit my blog site

\begin{center}
    \href{https://richie291.wixsite.com/theoreticalphysics}{https://richie291.wixsite.com/theoreticalphysics}
\end{center}

These notes are not endorsed by Professor Spannowsky or Durham University.

\vspace{1cm}

\begin{flushright}
    \Huge{{\cursive\setul{0.1ex}{}\ul{~~Richie Dadhley~~}}}
\end{flushright}

% -------------------------------------------------------------------
% Contents
% -------------------------------------------------------------------

\tableofcontents

% -------------------------------------------------------------------
% Main sections (as required)
% -------------------------------------------------------------------

\mainmatter

\chapter{Introduction}

Stuff to come. Thinking I might combine the notes on IFT, QFT1, QFT2, QED, QCD, SM etc into one big document. If I do do that, put an explanation here. If not just delete this chapter.
\input{sections/IFT1.intro.tex}
\chapter{Symmetries \& Hamiltonian Field Theory}

\section{Lorentz Invariance}

As we explained last lecture, we want to consider theories that are Lorentz invariant. Will therefore be testing for this often and so we want to clarify the notion and conventions here. 

As in (i) last lecture, we denote a general Lorentz transformation via the symbol $\Lambda$. We think of Lorentz transformations as acting on the spacetime coordinates $\{x^{\mu}\}$\footnote{In fact this interpretation is a bit misleading, but that's not important here. For an explanation of what I mean, see my \href{https://richie291.wixsite.com/theoreticalphysics/post/the-we-heraeus-international-winter-school-on-gravity-and-light}{notes on Dr. Schuller's Winter School on Gravity and Light}, Section 13.3.} and so it is useful to express them as matrices. We therefore have 
\bse 
    x^{\mu} \longrightarrow \widetilde{x}^{\mu} = {\Lambda^{\mu}}_{\nu} x^{\nu}
\ese 
as our transformation. We also have the condition 
\be 
\label{eqn:LorentzTransformationsOnMetric}
    {\Lambda^{\mu}}_{\tau} \eta^{\tau\sig} {\Lambda^{\nu}}_{\sig} = \eta^{\mu\nu}. 
\ee 

Lorentz transformations are essentially spatial rotations and boosts, and we can work out the entries of the matrix ${\Lambda^{\mu}}_{\nu}$ by consider the action on the $\{x^{\mu}\}$. We give an example of each below. 

\bex
    Let's set $(x^0,x^1,x^2,x^3)=(t,x,y,z)$. First let's consider a rotation around the $z$ axis. This does nothing to the $t$ axis and obviously doesn't effect $z$. We therefore just set the SO(2) rotation in the $xy$-plane by angle $\theta$. In other words we have 
    \bse 
        \begin{pmatrix}
            t \\
            x \\
            y \\
            z 
        \end{pmatrix} \longrightarrow \begin{pmatrix}
            t \\
            x\cos\theta - y\sin\theta \\
            y\sin\theta + x\cos\theta \\
            z 
        \end{pmatrix},
    \ese 
    and so we can conclude\footnote{In these notes I shall try and be consistent and put brackets around indexed objects to indicate that we mean the matrix.}
    \bse 
        ({\Lambda^{\mu}}_{\nu}) = \begin{pmatrix}
            1 & 0 & 0 & 0 \\
            0 & \cos\theta & -\sin\theta & 0 \\
            0 & \sin\theta & \cos\theta & 0 \\
            0 & 0 & 0 & 1
        \end{pmatrix}.
    \ese 
\eex

\bex 
    Following a similar idea to the previous example, we can show that for a boost along the $x$ axis we have 
    \bse 
        ({\Lambda^{\mu}}_{\nu}) = \begin{pmatrix}
            \g & -\g v & 0 & 0 \\
            -\g v & \g & 0 & 0 \\
            0 & 0 & 1 & 0 \\
            0 & 0 & 0 & 1
        \end{pmatrix},
    \ese 
    where $\g = 1/\sqrt{1-v^2}$.
\eex 

\subsection{Active vs. Passive Transformations}

There is a subtlety in taking transformations that can be wonderfully confusing the first time you hear it, and this is the difference between an active and passive transformation. This short section just aims to clear up the differences so that the notation that follows in these notes is not confusing. 

Let's say we want shift the temperature field in a room to the right. That is the temperature your left-hand neighbour is currently feeling you will feel after the transformation. There are essentially two ways to achieve this: firstly we could actually move the air particles themselves and so physically move the temperature to the right; secondly we could leave the air where it is and instead we, the people,\footnote{Not to be confused with \href{https://en.wikipedia.org/wiki/We_the_People_(disambiguation)}{this}.} could move to the left. In both cases you will end up feeling the temperature your left-hand neighbour previously felt, the question is "what's the difference?" 

The answer is that the former is an \textit{active} transformation whereas the latter is a \textit{passive} one. Put in a more mathematically meaningful way, an active transformation is one where the thing itself (in this case the temperature field) is moved, whereas a passive transformation is one where we simply shift the underlying reference frame/coordinate system (in this case, us). 

Those passionate about the concepts of relativity should now throw their arms up in protest. Why? Well because a passive transformation clearly depends on the choice of coordinate system and so should not be a physical thing. In contrast, the active transformation makes perfect sense without reference to any coordinate system. Put in terms of the example above, the passive transformation only makes sense if we are in the room and move ourselves and explain the shift, whereas the active transformation makes perfect sense even if we're not in the room at all. 

It is for this reason that we actually consider active transformations in field theory. That is we actually consider shifting the \textit{fields} in our problems and not the coordinate system. As we have just seen, when we \textit{measure} something (i.e. we personally measure the change in temperature), we can think of the active transformation as a passive one going in the other direction (i.e. we move left so that `the temperature moves right'). This is why in what is to follow we shall write the transformation on fields as follows
\be 
\label{eqn:ActiveFieldTransformation}
    \phi(x) \longrightarrow \widetilde{\phi}(x) = \phi\big(\Lambda^{-1}x\big).
\ee 

The above equation is actually a particular case for the action of the \textit{representation} of the Lorentz transformation on fields. In this case we are acting on a scalar field and so no $D[\Lambda]$ factors appear. For a general field we have 
\bse 
    \phi^{\mu\nu...}(x) \longrightarrow D[\Lambda]^{\mu}_{\sig} \big(D[\Lambda]^{\nu}_{\tau} ...\big) \phi^{\sig\tau...}\big(\Lambda^{-1}x\big).
\ese 

\section{Noether's Theorem}

\bt[Noether's Theorem]
\label{thrm:Noether}
    Every continuous symmetry of the Lagrangian gives rise to a conserved current, which we label $j^{\mu}(x)$, such that the equations of motion imply 
    \be 
    \label{eqn:pj=0}
        \p_{\mu}j^{\mu} = 0.
    \ee 
\et 

As we will see, symmetries and their conserved currents are incredibly important in field theory, and so Noether's theorem is an invaluable tool to field theorists. 

\bq 
    In order to prove Noether's theorem it is very helpful to work infinitesimally. As we are considering continuous symmetries, we can always do this. 
    
    Now the first thing we note is that the equations of motion, \Cref{eqn:EulerLagrangeDensity}, were obtained by varying the action and dropping surface terms. We therefore see that these will be completely unaffected for any transformation that takes the form 
    \be 
    \label{eqn:LagrangianBoundaryChange}
        \cL \longrightarrow \cL + \epsilon\p_{\mu}F^{\mu},
    \ee 
    where $F^{\mu}$ is some function of the fields and $\epsilon$ is some small parameter.
    
    Ok, so let's consider some infinitesimal transformation of the field
    \bse 
        \phi_a(x) \longrightarrow \phi_a(x) + \epsilon\del\phi_a(x).
    \ese 
    This gives a change in the action, of the general form\footnote{We use the fact that $\epsilon$ is a constant to take it outside the derivative.} 
    \bse 
        \begin{split}
            \epsilon\del\cL & = \frac{\p\cL}{\p\phi_a} \epsilon\del\phi_a + \frac{\p\cL}{\p(\p_{\mu}\phi_a)}\p_{\mu}(\epsilon\phi_a) \\
            & = \bigg[ \frac{\p\cL}{\p \phi_a} -\p_{\mu}\bigg(\frac{\p\cL}{\p(\p_{\mu}\phi_a)}\bigg)\bigg] \epsilon\del\phi_a + \epsilon\p_{\mu} \bigg(\frac{\p\cL}{\p(\p_{\mu}\phi_a)}\del\phi_a\bigg) \\
            & = \epsilon\p_{\mu} \bigg(\frac{\p\cL}{\p(\p_{\mu}\phi_a)}\del\phi_a\bigg),
        \end{split}
    \ese 
    where we have used \Cref{eqn:EulerLagrangeDensity} to set the square bracket term to zero. Now if this transformation is a symmetry of the system then we also have 
    \bse 
        \epsilon\del\cL = \p_{\mu}F^{\mu},
    \ese
    and so taking these two results away from each other gives us 
    \mybox{
        \be 
        \label{eqn:NoetherCurrent}
            j^{\mu} = \frac{\p\cL}{\p(\p_{\mu}\phi_a)}\del\phi_a - F^{\mu}(\phi_a) \qquad \implies \qquad \p_{\mu}j^{\mu} = 0.
        \ee 
    }
    So this is our conserved current. We get one of these for each continuous $\del\phi$ transformation and so we have proved Noether's theorem. 
\eq 

Before doing some examples, first there is an important comment to make. As we have been careful to say above, Noether's theorem tells us that we get a conserved \textit{current}. This is a stronger result then saying we get a conserved \textit{charge}. Indeed a conserved current gives rise to a conserved charge, and we get this charge as follows: 
\bse 
    \p_{\mu}j^{\mu} = 0 \qquad \implies \qquad \frac{\p j^0}{\p t} = - \nabla \cdot \Vec{j}.
\ese 
Integrating both sides of this over all of space, we have 
\bse 
    \begin{split}
        \frac{\p}{\p t} \int_{\R^3} d^3 \vec{x} \, j^0 & = -\int_{\R^3}d^3\vec{x} \, \nabla\cdot\vec{j} \\
        & = - \oint d\vec{s} \cdot \vec{j} \\
        & = 0,
    \end{split}
\ese 
where the last line follows from the fact that we assume $\vec{j}$ falls off sufficiently fast as $|\vec{x}|\to\infty$. We can therefore define our conserved charge by 
\be 
\label{eqn:NoetherCharge}
    Q := \int_{\R^3} d^3 \vec{x} \, j^0.
\ee 
This is a \textit{globally} conserved charge. 

The reason that the existence of a conserved current is stronger than the existence of a globally conserved charge is that the current tells us that charge is conserved \textit{locally}. This is easily seen by repeating the above procedure but now just integrating over some finite volume, $V$. We obtain 
\bse 
    \frac{dQ_V}{dt} = - \oint_{\p V} \vec{j}\cdot d\vec{S},
\ese 
where $\p V$ is the boundary of $V$. This equation tells us the the charge leaving a volume $V$ in time $t$ (left-hand side) is equal to the current flowing through the boundary of the volume (right-hand side). 

This is a much more powerful statement than simply that charge is conserved globally. For further clarity, if charge is conserved locally everywhere then we know that it is conserved globally (simply take your local region to be infinitely big). However if we only know that charge is conserved globally it could be possible that a charge disappears at some point $x\in\R^3$ and miraculously reappears at some other point $y\in\R^3$ far away from $x$. In other words the charge `teleported' through space. 

\subsection{The Energy-Momentum Tensor}

An important example of Noether's theorem comes from considering translations in spacetime. Recall that in particle mechanics, translations in space gave rise to momentum conservation, whereas translations in time gave rise to energy conservation. We are now dealing with a relativistic theory and so want to avoid this splitting of spacetime as much as possible, and so we want to find a 4-dimensional generalisation of the above conservation laws. This is exactly the \textit{energy-momentum tensor}. We denote it by a $T$. 

\br 
    Some people also refer to the energy-momentum tensor as the stress-energy tensor or the stress-momentum tensor, or even the stress-energy-momentum tensor. I will try to use energy-momentum tensor everywhere here, but I might sometimes just call it the stress tensor. Apologies for this in advance.
\er 

Ok, let's derive this lovely chappy.

Consider a continuous translation of the spacetime coordinates. As above, we shall work infinitesimally, and so we have 
\bse 
    x^{\mu} \longrightarrow x^{\mu} - \epsilon^{\mu},
\ese 
where we use a minus sign so that the transformation of the field is positive, see the discussion of active vs. passive transformations above. We can Taylor expand the field to obtain the transformation
\bse 
    \phi_a(x) \longrightarrow \phi_a(x) + \epsilon^{\mu}\p_{\mu}\phi_a(x),
\ese 
and similarly the Lagrangian transforms as 
\bse 
    \cL(x) \longrightarrow \cL(x) + \epsilon^{\mu}\p_{\mu}\cL(x).
\ese 
Comparing this to \Cref{eqn:LagrangianBoundaryChange}, we see that $F^{\mu}=\epsilon^{\mu}\cL$. Our conserved current is therefore 
\bse
    \begin{split}
        j^{\mu} & = \frac{\p \cL}{\p(\p_{\mu}\phi_a)} \epsilon^{\nu}\p_{\nu}\phi_a - \epsilon^{\mu}\cL \\
        & = \epsilon^{\nu}\bigg(\frac{\p \cL}{\p(\p_{\mu}\phi_a)} \p_{\nu}\phi_a - \del^{\mu}_{\nu}\cL\bigg),
    \end{split}
\ese 
and so we have a conserved current for each $\epsilon^{\nu}$, i.e. for the translations in each direction. These are the components of our energy-momentum tensor.
\mybox{
\be
\label{eqn:EnergyMomentumTensor}
    {T^{\mu}}_{\nu} := (j^{\mu})_{\nu} = \frac{\p \cL}{\p(\p_{\mu}\phi_a)} \p_{\nu}\phi_a - \del^{\mu}_{\nu}\cL.
\ee 
} 
Being a conserved current it satisfies 
\be 
\label{eqn:EnergyMomentumConservation}
    \p_{\mu}{T^{\mu}}_{\nu} = 0 \qquad \forall \nu = 0,1,...,d.
\ee 

In spacetime (i.e. $d=4$) we have four currents and so we have 4 charges. The $\nu$ index above tells us which direction we translated in, and so in correspondance with the comment made at the start of this section we want $\nu=0$ to correspond to energy conservation and $\nu=1,2,3$ to correspond to momentum conservation. We therefore define\footnote{Note we have raised the $\nu$ index here. As we are working in Minkowksi spacetime this really isn't a big deal, however as these are definitions you should be careful in more general spacetimes as some non-trivial factors will appear.}
\be 
\label{eqn:EnergyAndMomentumCharges}
    E := \int d^3\vec{x} \, T^{00}, \qand P^i := \int d^3\, \vec{x}T^{0i}
\ee 
to be the total energy and the ($i$-th component of the) total momentum of the field configuration, respectively. 

\br 
    Note in deriving \Cref{eqn:EnergyMomentumTensor}, we didn't say anything about the actual form of our Lagrangian, and so this result holds for generic $\cL$.
\er 

\subsection{Angular Momentum \& Boost Symmetries}

So we have done spacetime translations, but recall that we have also restricted ourselves to Lagrangians that are Lorentz invariant. Given that Lorentz transformations are continuous, Noether's theorem tells us that we should have some conserved currents to go along with them. Keeping in line with the logic applied to the translations, what kind of conservation do we expect? Well in particle mechanics spatial rotations give rise to conservation of angular momentum, so we expect some generalisation of this. As we are considering the whole set of Lorentz transformations, we will also derive the symmetries corresponding to boosts, whatever these may be. 

As always, we want to work infinitesimally, and its a fact that infinitesimal Lorentz transformations can be written as
\bse 
    {\Lambda^{\mu}}_{\nu} = {\del^{\mu}}_{\nu} + {\omega^{\mu}}_{\nu},
\ese 
for some infintesimal ${\omega^{\mu}}_{\nu}$, and where \Cref{eqn:LorentzTransformationsOnMetric} tells us 
\be 
\label{eqn:AntisymmetricOmega}
    \omega^{\mu\nu} = - \omega^{\nu\mu}.
\ee 

From \Cref{eqn:ActiveFieldTransformation}, we have, after Taylor expanding
\bse 
    \phi(x) \longrightarrow \phi(x^{\mu}) - {\omega^{\mu}}_{\nu} x^{\nu} \p_{\mu}\phi(x) \qquad \implies \qquad \del\phi = -{\omega^{\mu}}_{\nu}x^{\nu}\p_{\mu}\phi.
\ese 
Similarly we obtain 
\bse 
    \del\cL = -{\omega^{\mu}}_{\nu} x^{\nu} \p_{\mu}\cL.
\ese 
Now we use a clever trick. Firstly we note that ${\omega^{\mu}}_{\nu}$ is a constant so we can take it inside the derivative. Next we note that 
\bse 
    \p_{\mu}x^{\nu} = \del^{\nu}_{\mu},
\ese 
along with ${\omega^{\mu}}_{\mu}=0$ by \Cref{eqn:AntisymmetricOmega}. We can therefore also take the $x^{\nu}$ inside the derivative and obtain 
\bse 
    \del\cL = -\p_{\mu}\big( {\omega^{\mu}}_{\nu} x^{\nu}\cL\big).
\ese 
This is now of the form \Cref{eqn:LagrangianBoundaryChange}, with $F^{\mu} = {\omega^{\mu}}_{\nu} x^{\nu}\cL$. So our conserved currents are 
\bse
    \begin{split}
        j^{\mu} & = - \frac{\p \cL}{\p(\p_{\mu}\phi)}{\omega^{\sig}}_{\nu}x^{\nu}\p_{\sig}\phi + {\omega^{\mu}}_{\nu} x^{\nu}\cL \\
        & = - {\omega^{\sig}}_{\nu} \bigg[ \frac{\p \cL}{\p(\p_{\mu}\phi)}\p_{\sig}\phi + {\del^{\mu}}_{\sig} \cL \bigg]x^{\nu} \\
        & = - {\omega^{\sig}}_{\nu} {T^{\mu}}_{\sig} x^{\nu}.
    \end{split}
\ese 

As above we can split this into the individual currents, one for each ${\omega^{\sig}}_{\nu}$, and obtain 
\be 
\label{eqn:LorentzCurrents}
    (\cJ^{\mu})^{\rho\sig} = x^{\rho}T^{\mu\sig} - x^{\sig}T^{\mu\rho}, \qquad \text{with} \qquad \p_{\mu}(\cJ^{\mu})^{\rho\sig} = 0 \qquad \forall \rho,\sig=0,..,3.
\ee 
Each one of these has a corresponding conserved charge. We get the particle result, i.e. angular momentum, when $\rho,\sig=1,2,3$: 
\bse 
    Q^{ij} = \int d^3\vec{x} \, \big(x^iT^{0j} - x^jT^{0i}\big).
\ese 
The question is "what charge do the boosts give us?" Well these correspond to $\rho=0, \sig=1,2,3$, which give 
\bse 
    Q^{0i} = \int d^3\vec{x} \, \big(x^0T^{0i} - x^iT^{00}\big).
\ese 
What is this? Well we know that its temporal derivative vanishes, so using $x^0=t$, we have 
\bse 
    \begin{split}
        \frac{d Q^{0i}}{dt} & = \int d^3\vec{x} \, T^{0i} + \int d^3\vec{x} \, t \frac{\p T^{0i}}{\p t} - \frac{d}{dt}\int d^3\vec{x} \, x^i T^{00} \\
        0 & = P^i + \frac{d P^i}{dt} - \frac{d}{dt}\int d^3\vec{x} \, x^i T^{00},
    \end{split}
\ese 
where we've used \Cref{eqn:EnergyAndMomentumCharges}. But we've already seen that $P^i$ is a constant and so we simply get 
\bse 
    \frac{d}{dt}\int d^3\vec{x} \, x^i T^{00} = \text{constant},
\ese 
which is the statement that the center of energy of the field travels with constant velocity. 

\br 
    Note that \Cref{eqn:AntisymmetricOmega} agrees with the number of Lorentz transformations. That is, it is an antisymmetric, $4\times4$ matrix and so has $4\times 3/2=6$ independant entries. So it has $6$ basis elements, $3$ of these correspond to spatial rotations and the other $3$ are our boosts.
\er 

\subsection{Internal Symmetries}

In the above calculations we considered transformations that did something to the spacetime itself, that is the $x^{\mu}$s changed. The question is "is this the only kind of symmetry we can have?" The answer is no, but in order to see it we need to have at least two fields in our Lagrangian. When this is the case we can consider transformations in the plane spanned by these two fields. If our transformations give rise to a symmetry, then we call them \textit{internal symmetries}. Let's look at some examples. 

\bex 
    Consider a theory with 2 real scalar fields $\phi_1$ and $\phi_2$. We can package these together into a two column representation simply as 
    \bse 
        \vec{\phi} := \begin{pmatrix}
            \phi_1 \\
            \phi_2
        \end{pmatrix}.
    \ese 
    We can consider these spanning some 2-dimensional plane as indicated diagrammatically below. Let's then consider the Lagrangian 
    \bse 
        \begin{split}
            \cL & = \frac{1}{2}(\p_{\mu}\vec{\phi})\cdot(\p^{\mu}\phi) - \frac{1}{2}m^2 \vec{\phi}\cdot \vec{\phi} \\
            & = \frac{1}{2}\big[(\p\phi_1)^2 + (\p\phi_2)^2\big] - \frac{1}{2}m^2 \big[ \phi_1^2 + \phi_2^2 \big]
        \end{split}
    \ese 
    where on the first line we have written out the multiple of the derivatives explicitly to make the dot-product clear. This Lagrangian is invariant under rotations in the $\phi_1-\phi_2$ plane, i.e. under 
    \bse 
        \begin{split}
            \phi_1 \longrightarrow \phi_1' & = \phi_1\cos\theta + \phi_2\sin\theta \\
            \phi_2 \longrightarrow \phi_2' & = - \phi_1\sin\theta + \phi_2\cos\theta.
        \end{split}
    \ese 
    This is a continuous transformation and so we can work infinitesimally and find the conserved current. We have 
    \bse 
        \begin{split}
            \phi_1' & \approx \phi_1 + \theta \phi_2 \qquad \implies \qquad \del\phi_1 = \theta\phi_2 \\
            \phi_1' & \approx -\theta\phi_1 + \phi_2 \qquad \implies \qquad \del\phi_2 = -\theta\phi_1.
        \end{split}
    \ese 
    
    \begin{center}
        \btik 
            \draw[thick] (-1,0) -- (3,0);
            \node at (3,-0.3) {\large{$\phi_1$}};
            \draw[thick] (0,-1) -- (0,3);
            \node at (-0.3,3) {\large{$\phi_2$}};
            \draw[thick, dashed, rotate around={20:(0,0)}] (0,0) -- (3,0);
            \node at (3.2,1) {\large{$\phi_1'$}};
            \node at (1.2,0.2) {\large{$\theta$}};
            \draw[thick, dashed, rotate around={20:(0,0)}] (0,0) -- (0,3);
            \node at (-1.3,2.7) {\large{$\phi_2'$}};
            \node at (-0.15,1) {\large{$\theta$}};
        \etik 
    \end{center}
    
    The Lagrangian doesn't change, so our conserved current is simply 
    \bse 
        \begin{split}
            j^{\mu} & = \frac{\p \cL}{\p (\p_{\mu}\phi_1)} \del\phi_1 + \frac{\p \cL}{\p (\p_{\mu}\phi_2)} \del\phi_2 \\
            & = (\p^{\mu}\phi_1)\phi_2 - (\p^{\mu}\phi_2)\phi_1.
        \end{split}
    \ese 
    We can check that this obeys $\p_{\mu}j^{\mu}=0$. First we need the Euler-Lagrange equations for our action. We have 
    \bse 
        0 = \p_{\mu}\bigg(\frac{\p\cL}{\p(\p_{\mu}\phi_1)}\bigg) - \frac{\p \cL}{\p \phi_1} = (\p^2 + m^2)\phi_1
    \ese
    and similarly for $\phi_2$. Now take the derivative of our current:
    \bse
        \begin{split}
            \p_{\mu}j^{\mu} & = (\p^2\phi_1)\phi_2 + \p^{\mu}\phi_1\p_{\mu}\phi_2 - (\p^2\phi_2)\phi_1 - \p^{\mu}\phi_2\p_{\mu}\phi_1 \\
            & = m^2\phi_1\phi_2 - m^2\phi_2\phi_1 \\
            & = 0,
        \end{split}
    \ese 
    where we used the Euler-Lagrange equations to get to the third line, and then used the fact that we have real scalars so $\phi_1\phi_2=\phi_2\phi_1$. 
    
    This symmetry is known as a \textit{global SO(2) symmetry}. The name makes sense: its global (i.e. the whole plane is rotated) and its a rotation in 2-dimensions. It is actually a specific case of the more general global SO($N$) symmetry, which has 
    \bse 
        \frac{N(N-1)}{2}
    \ese 
    conserved currents for $N$ fields. 
\eex 

\bex 
    We can reformulate the previous example by considering the complex scalar field
    \bse 
        \psi = \frac{1}{\sqrt{2}}\big( \phi_1 + i\phi_2\big).
    \ese 
    The above Lagrangian then becomes\footnote{Note that we don't have a factor of $1/2$ in this expression as was the case for the real scalar field. This is just because of the $1/\sqrt{2}$ factor above.} 
    \be 
    \label{eqn:ComplexLagrangianClassical}
        \cL = \p_{\mu}\psi^*\p^{\mu}\psi - m^2 \psi^*\psi.
    \ee 
    We are now rotating in the complex plane so instead of considering a SO(2) rotation we consider the complex U(1) rotation
    \bse 
        \psi \longrightarrow e^{i\a}\psi.
    \ese 
    We already know that this is a symmetry because its exactly the same as the previous example, but what does the current look like in this complexified case? Well simple calculation will give 
    \be 
    \label{eqn:ComplexCurrent}
        j^{\mu} = i\big[ (\p^{\mu}\psi^*)\psi - \psi^*(\p^{\mu}\psi)\big]
    \ee 
\eex 

\bbox 
    Derive the above conserved current. \textit{Hint: Note that $\psi^* \longrightarrow e^{-i\a}\psi^*$ and then work infinitesimally.} 
\ebox 

\br 
    \textcolor{red}{Maybe put a comment here in line with the non-abelian comment made in Tong, page 18.}
\er 

Internal symmetries will prove to be vital for the study of particle physics as a QFT. We will see that the charges arising from such symmetries correspond to things such as electric charge and particle number. 

\section{Hamiltonian Field Theory}

Above we have constructed classical field theory in terms of Lagrangians. It is true that we can extend this Lagrangian approach to QFT by using so-called \textit{path integrals}, and there are advantages to doing that (most notably that the Lorentz invariance is manifest throughout), however in this course we shall use the Hamiltonian approach instead. This uses so-called \textit{canonical quantisation} to promote the fields to operators. 

In particle mechanics, we define the Hamiltonian to be 
\bse 
    H(p,q) = \sum_a p_a\dot{q}_a - L(q,\dot{q}),
\ese
where 
\bse 
    p_a := \frac{\p L}{\p \dot{q}_a}
\ese 
is known as the \textit{conjugate momentum}. The idea is to eliminate $\dot{q}$ wherever we can and replace it with $p$. 

In field theory we do a very similar thing, but now we have to use the Lagrangian density, $\cL$, and we define the \textit{conjugate momentum density} and \textit{Hamiltonian density}
\mybox{
\be 
\label{eqn:ConjugateMomentumDensity}
    \pi(x) := \frac{\p \cL}{\p \dot{\phi}(x)}
\ee 
\be 
\label{eqn:HamiltonianDensity}
    \cH = \pi(x)\dot{\phi}(x) - \cL(x).
\ee 
}
Again we favour replacing any $\dot{\phi}(x)$ dependence with $\pi(x)$ once we know the relation. As with \Cref{eqn:LtocL}, we define the Hamiltonian as the spatial integral over the Hamiltonian density, 
\be 
    H(t) = \int d^3\vec{x} \, \cH(x).
\ee 

Note in this line we appear to have broken Lorentz symmetry as we have picked a preferred time to define our Hamiltonian. However, as long as we are careful not to do anything silly, we should be alright because we started with a Lorentz invariant theory. This is what we mean by the Lorentz symmetry not being manifest: it is not obvious just by looking at the equations that we have Lorentz symmetry, in contrast to the Lagrangian formalism where spacetime indices were always summed over and so it was clear.

\bex 
    As an example let's consider the real scalar field mentioned above with Lagrangian\footnote{It is incredibly common in the field theory world to forget to say `density', as I have done here (and probably above too). The symbols should tell you what we mean: densities are normally given in fancy curly font.}
    \bse 
        \begin{split}
            \cL & = \frac{1}{2}(\p\phi_a)^2 - \frac{1}{2}m^2 \phi_a^2 \\
            & = \frac{1}{2}\dot{\phi}_a^2 - \frac{1}{2}(\nabla\phi_a)^2 - \frac{1}{2}m^2 \phi_a^2.
        \end{split}
    \ese 
    The conjugate momentum is therefore 
    \bse 
        \pi_a(x) = \dot{\phi}_a(x),
    \ese 
    and the Hamiltonian density is 
    \bse
        \begin{split}
            \cH & = \dot{\phi}_a^2 - \cL \\
            & = \frac{1}{2}\dot{\phi}_a^2 + \frac{1}{2}(\nabla\phi_a)^2 + \frac{1}{2}m^2\phi_a^2 \\
            & = \frac{1}{2}\pi_a^2 + \frac{1}{2}(\nabla\phi_a)^2 + \frac{1}{2}m^2\phi_a^2,
        \end{split}
    \ese 
    where in the last line we have done our procedure of swapping out the $\dot{\phi}$s for $\pi$s using the relation above.\footnote{For clarity, you don't just swap $\dot{\phi}\longrightarrow\pi$, but you use the relation for $\pi$ in terms of $\dot{\phi}$ to eliminate the $\dot{\phi}$s. In this particular case, $\pi=\dot{\phi}$, and it does correspond to just swapping them.} The integral over these terms give three contributions to the Hamiltonian, they are 
    \bse 
        H = \int d^3 \vec{x} \, \bigg(\underbrace{\frac{1}{2}\pi_a^2}_{\text{kinetic}} + \underbrace{\frac{1}{2}(\nabla\phi_a)^2}_{\text{shear}} + \underbrace{\frac{1}{2}\phi_a^2}_{\text{mass}}\bigg).
    \ese 
\eex
\chapter{Canonical Quantisation \& Free Klein Gordan Field}

\section{Canonical Quantisation}

Recall the idea behind quantisation in QM is to promote the generalised coordinates in the classical theory and `promote' them to operators acting on the Hilbert space of the quantum theory. This recipe is called \textit{canonical quantisation}. The Poisson bracket\footnote{\textcolor{red}{Note to self: maybe put something about Poisson brackets above when discussing generalised coordinates.}} relation between the generalised coordinates magically transformed into a commutation relation between the operators. That is\footnote{There are factors of $\hbar$s in these equations, but in natural units they go bye-bye.} 
\begin{equation*}
    \begin{split}
        \{q_a,a_b\} = \{p^a,p^b\} = 0 \qquad & \longrightarrow \qquad [\hat{q}_a,\hat{q}_b] = [\hat{p}^a,\hat{p}^b] = 0 \\
        \{q_a,p^b\} = \del^b_a \qquad & \longrightarrow \qquad [\hat{q}_a,\hat{p}^b] = i\del^b_a
    \end{split}
\end{equation*} 
As we are working in Minkowski spacetime, we can lower the indices on the $p$s easily and obtain 
\bse 
    [\hat{q}_a,\hat{q}_b] = [\hat{p}_a,\hat{p}_b] = 0 \qand [\hat{q}_a,\hat{p}_b] = i\del_{ab}.
\ese 

We adopt the same philosophy for fields, and promote them to \textit{operator valued functions}. However we have a small problem: for the particle mechanics case we just had a finite number of generalised coordinates and so it was easy to do, whereas for the fields there's an infinite number, one for each point $\Vec{x}$ in space. We treat this spatial dependence as a label and so need something analogous to the $\del_{ab}$ above. The answer is obviously the usual delta function.  We use the Schr\"{o}dinger picture, where all time dependence appears in the states $\ket{\psi}$ which obey the Schr\"{o}dinger equation
\bse 
    i\frac{d\ket{\psi}}{dt} = H\ket{\psi},
\ese 
and the operators themselves are time independent, and write\footnote{You can work in the Heisenberg picture and define what are known as `equal time' commutation relations.}
\be 
\label{eqn:FieldsCommutation}
    [\phi_a(\Vec{x}),\phi_b(\Vec{y})] = [\pi_a(\Vec{x}),\pi_b(\Vec{y})] = 0, \qand [\phi_a(\Vec{x}), \pi_b(\Vec{y})] = i \del^{(3)}(\Vec{x}-\Vec{y}) \del_{ab}.
\ee 

\br 
\label{rem:ButcheredLorentzInv}
    Note we have really butchered our manifest Lorentz invariance here: we completely separated space and time and made the operators only functions of space! Of course it must be true that we're alright to do this and still get a Lorentz invariant theory, otherwise we wouldn't be doing it in these notes. However this choice of doing things will introduce some factors here and there (e.g. to the measures in integrals) to ensure Lorentz invariance.
\er 

\section{Free Theories}

Recall that the typical goal of QM is to find the spectrum (i.e. eigenvalues) of operators, in particular the Hamiltonian. We now want to do a similar thing for QFTs, however this turns out to be an incredibly hard thing to do, as we now have an uncountably infinite number of degrees of freedom (one for each $\Vec{x}$ value)! The question is "can we somehow get around this?" The answer is "yes, but it restricts the type of theories we consider." What we do is consider theories in which each degree of freedom evolved independently to all others. This is essentially the statement that $\phi(\vec{x})$ and $\phi(\vec{y})$ don't talk to each other unless $\vec{x}=\vec{y}$. For a reasonably self explanatory reason, we call these theories \textit{free theories}.

\br
    Of course free theories are boring from a physical perspective (as nothing interacts so there are essentially no forces!), and we want to study interacting theories. We will return to these later, and study them as perturbations using Feynman diagrams, but first we need to develop the mathematical tools of free theory. So hold tight, more interesting stuff is coming. 
\er 

Ok, so how do we go about quantising the free fields and finding their spectrum? Well the answer is to consider our lovely friend the classical free Klein-Gordan field. Recall that the Euler-Lagrange equations gave us the Klein-Gordan equation, \Cref{eqn:ClassicalKleinGordan}:
\bse 
    (\p^2 + m^2)\phi = 0.
\ese 
As we are working in Minkowski spacetime, we have a global notion of time\footnote{Again this comment is just made because things are different in general spacetimes, see footnote 3 from the first lecture.} and so we can decompose these fields in terms of their Fourier transform
\bse 
    \begin{split}
        \phi(\vec{x},t) & = \int \frac{d^3\vec{p}}{(2\pi)^3} \, \frac{1}{2}\Big[  \widetilde{\phi}(\vec{p},t)e^{i\vec{p}\cdot\vec{x}} + \widetilde{\phi}^*(\vec{p},t)e^{-i\vec{p}\cdot\vec{x}}\Big] \\
        & = \int \frac{d^3\vec{p}}{(2\pi)^3} \, \widetilde{\phi}(\vec{p},t)e^{i\vec{p}\cdot\vec{x}},
    \end{split}
\ese 
where to get to the second line we have identified $\widetilde{\phi}^*(-\vec{p},t) = \widetilde{\phi}(\vec{p},t)$. Using this coordinate system, our Klein-Gordan equation reduces to (dropping the tilde and using the argument to indicate which function it is)
\bse 
    \bigg( \frac{\p^2}{\p t^2} + (\vec{p}+m^2)\bigg)\phi(\vec{p},t) = 0.
\ese

Then the keen-eyed person notes that \textit{for each value} of $\vec{p}$ this corresponds to its own harmonic oscillator with frequency 
\be
\label{eqn:KGHarmonicFrequency}
    \omega_{\vec{p}} = \sqrt{\vec{p}^2+m^2}.
\ee 
To stress the point, we get a harmonic oscillator each \textit{every single} value of $\vec{p}$ independently from any other value. The general field $\phi(\vec{x},t)$ is then simply given by a linear superposition of (an infinite number of) harmonic oscillators. We therefore have, at least classically, achieved the goal above to frame the theory in such a way that each degree of freedom (each $\vec{p}$) evolves independently from the others. Now what we want to do is quantise it. 

\br 
    Note that in order to take the Fourier expansion above we need to assume that the field $\phi(\vec{x},t)$ die off sufficiently quickly as $|\vec{x}|\to \infty$. In order to guarantee this, we use fields from the so-called \textit{Schwartz space}. A precise definition (with a lot more context for why they're useful in QM) can be found in Simon Rea and my \href{https://richie291.wixsite.com/theoreticalphysics/post/dr-frederic-schuller-s-course-of-quantum-theory}{notes on Dr. Schuller's course on quantum theory}, but for here we shall just say they are functions that die off at infinity as do their derivatives. 
\er 

\subsection{A Quantum Field Theorist's Best Friend: The Harmonic Oscillator}

\bnn 
    In this section, and most likely in everything to follow from now on, I am going to drop the hats on quantum operators. I might reinstate them at some points for clarity, but we shall see. 
\enn 

As we have been doing above, let's forget about field theory for a minute and study regular old quantum mechanics. Recall that the quantum harmonic oscillator (QHO) has the Hamiltonian (in 1-dimension)
\be
\label{eqn:QHOHamiltonian}
    H = \frac{1}{2}p^2 + \frac{1}{2}\omega^2q^2,
\ee 
where $p$ and $q$ are the canonical operators obeying 
\be 
\label{eqn:pqcommutator}
    [q,p]=i.
\ee 
We can rewrite this Hamiltonian in a nicer form by introducing the \textit{creation} and \textit{annihilation operators}: $a^{\dagger}$ and $a$, respectively. These are defined such that 
\be 
\label{eqn:qpcreationannihilation}
    q = \frac{1}{\sqrt{2\omega}}\big( a + a^{\dagger} \big), \qand p = -i\sqrt{\frac{\omega}{2}}\big( a - a^{\dagger} \big).
\ee 

\bbox
    Show that \Cref{eqn:pqcommutator} and \Cref{eqn:qpcreationannihilation} give 
    \be 
    \label{eqn:creationqpcreationannihilationcommutator}
        [a,a^{\dagger}] = 1.
    \ee 
    Then using this result show that the Hamiltonian \Cref{eqn:QHOHamiltonian} can be rewritten as 
    \be
    \label{eqn:QHOHamiltonianaadagger}
        H = \omega\bigg(a^{\dagger}a + \frac{1}{2}\bigg).
    \ee 
    Finally show that 
    \be 
    \label{eqn:HaadaggerCommutation}
        [H,a^{\dagger}] = \omega a^{\dagger}, \qand [H,a] = -\omega a.
    \ee 
\ebox

We can now begin to look at the spectrum of the QHO. The \textit{ground state} $\ket{0}$ is defined via 
\be 
\label{eqn:QHOGroundState}
    a\ket{0} = 0
\ee 
and so we see from \Cref{eqn:QHOHamiltonianaadagger} that the ground state has energy 
\bse 
    E_0 = \frac{\omega}{2}.
\ese 
Now using \Cref{eqn:HaadaggerCommutation}, we see that 
\bse 
    \begin{split}
        H(a^{\dagger}\ket{0}) & = a^{\dagger}(H\ket{0}) + \omega (a^{\dagger}\ket{0}) \\
        & = (E_0+\omega)(a^{\dagger}\ket{0}),
    \end{split}
\ese 
and so we see that $a^{\dagger}\ket{0}$ is an eigenvector of $H$ with energy $E=E_0+\omega$. We therefore define \textit{excited states} as\footnote{We're ignoring normalisation here, i.e. $\braket{n}{n}\neq 1$. This is not important for our present discussion.} 
\be 
\label{eqn:QHOExcitedStates}
    \ket{n} := (a^{\dagger})^n\ket{0}.
\ee 
Extending the calculation above, these states have energy 
\be 
\label{eqn:QHOEn}
    E_n = \bigg(n+\frac{1}{2}\bigg)\omega,
\ee 
and so we build a spectrum for our theory as integer steps in $\omega$, as indicated pictorially below. 
\begin{center}
    \btik 
        \draw[thick] (0,0) -- (3,0);
        \node at (-0.5,0) {\large{$\ket{0}$}};
        \node at (3.5,0) {\large{$\frac{1}{2}\omega$}};
        \draw[thick] (0,1) -- (3,1);
        \node at (-0.5,1) {\large{$\ket{1}$}};
        \node at (3.5,1) {\large{$\frac{3}{2}\omega$}};
        \draw[thick] (0,2) -- (3,2);
        \node at (-0.5,2) {\large{$\ket{2}$}};
        \node at (3.5,2) {\large{$\frac{5}{2}\omega$}};
        \node at (1.5,2.75) {\Huge{$\mathbf{\vdots}$}};
    \etik 
\end{center}

As the notation suggests (and as can easily be seen from \Cref{eqn:qpcreationannihilation}) the creation and annihilation operators are Hermitian conjugates to each other. This translates into bra-ket notation in terms of left and right actions: 
\be 
\label{eqn:HermitionLeftRightAction}
    \big(a\ket{\psi}\big)^{\dagger} = \bra{\psi}a^{\dagger}.
\ee 

\subsection{Spectrum Of Quantum Klein-Gordan Field}

Ok so we now know how to get the spectrum of a single QHO. We have also seen that the Klein-Gordan field is essentially an infinite sum (i.e. an integral) of QHOs, so now all we need to do to get the spectrum of the Hamiltonian is integrate over an infinite number of creation and annihilation operators, labelled by $\vec{p}$ --- $a^{\dagger}_{\vec{p}}$ and $a_{\vec{p}}$. To do this we express $\phi$ and $\pi$ as
\be 
\label{eqn:phipicreationannihilation}
    \begin{split}
        \phi(\vec{x}) & = \int \frac{d^3\vec{p}}{(2\pi)^3} \frac{1}{\sqrt{2\omega_{\vec{p}}}} \Big[ a_{\vec{p}} \, e^{i\vec{p}\cdot\vec{x}} + a^{\dagger}_{\vec{p}} \, e^{-i\vec{p}\cdot\vec{x}}\Big] \\
        \pi(\vec{x}) & = \int \frac{d^3\vec{p}}{(2\pi)^3} (-i)\sqrt{\frac{\omega_{\vec{p}}}{2}} \Big[ a_{\vec{p}} \, e^{i\vec{p}\cdot\vec{x}} - a^{\dagger}_{\vec{p}} \, e^{-i\vec{p}\cdot\vec{x}}\Big],
    \end{split}
\ee 
where the form of these expressions makes sense when comparing to \Cref{eqn:qpcreationannihilation}.

\br 
    Note that it is only at this point that we are now considering the \textit{quantum} Klein-Gordan field. This is in contrast why I was careful to emphasise before that we were considering the classical Klein-Gordan field. The equations of motion are still the same, but now they are quantum expressions. 
\er 

This looks nice, but what about the commutators? 
\bcl 
\label{claim:FieldCommutator}
    \Cref{eqn:FieldsCommutation} hold, if and only if, 
    \be 
    \label{eqn:FieldCreationAnnihilationCommutator}
        [a_{\vec{p}} \, , a_{\vec{q}}] = \Big[a^{\dagger}_{\vec{p}} \, , a^{\dagger}_{\vec{q}}\Big] = 0, \qand \Big[a_{\vec{p}} \, , a^{\dagger}_{\vec{q}}\Big] = (2\pi)^3 \del^{(3)}(\vec{p}-\vec{q}).
    \ee 
\ecl 
This claim is not complicated to prove, but a bit of a pain to write out and so I, lovingly, decided to set them as an exercise.

\bbox 
    Prove \Cref{claim:FieldCommutator}. Note it is an if and only if statement so you need to show it both ways, i.e. the $\phi/\pi$ commutators imply the $a/a^{\dagger}$ ones and visa versa. \textit{Hint: If you get very stuck, Prof. Tong sketches one on page 24 of his notes.}
\ebox 

\br
\label{rem:Sandwiching}
    \Cref{eqn:FieldCreationAnnihilationCommutator} seem strange: the left-hand side is a commutator between operators whereas the right-hand side is a delta distribution? The way we wrap our heads around this is to remember that any physical results in QM appear in inner products $\bra{\psi}A\ket{\psi}$, which can be written as integrals and so the delta function makes sense. 
    
    Using \Cref{eqn:HermitionLeftRightAction}, and the fact that our states are orthogonal, we see that it is only the terms that contain $a_{\vec{p}}\, a^{\dagger}_{\vec{p}}$ or $a^{\dagger}_{\vec{p}} \,  a_{\vec{p}}$ that contribute to expectation values. We will see an example of this below when finding the momentum of the first excited state. 
    
    This idea, and similar ones where we get $\C$-numbers on the right-hand side, comes up again and again in QFT. What we have to keep telling ourselves is that "remember we're sandwiching this between states!" and then go from there.
\er 

Ok, so we have already derived that the Hamiltonian for this system is 
\bse 
    H = \frac{1}{2}\int d^3\vec{x} \, \big(\pi^2 + (\nabla\phi)^2 + m^2\phi^2\big).
\ese 
We can substitute \Cref{eqn:phipicreationannihilation} in and obtain 
\be 
\label{eqn:KGHamiltonian}
    \begin{split}
        H & = \int \frac{d^3\vec{p}}{(2\pi)^3} \frac{\omega_{\vec{p}}}{2}\Big[ a_{\vec{p}} \, a^{\dagger}_{\vec{p}} + a^{\dagger}_{\vec{p}} \, a_{\vec{p}}\Big] \\
        & = \int \frac{d^3\vec{p}}{(2\pi)^3} \, \omega_{\vec{p}} \bigg[ a^{\dagger}_{\vec{p}}\, a_{\vec{p}} + \frac{1}{2}(2\pi)^3\del^{(3)}(0)\bigg],
    \end{split}
\ee 
where to get to the last line we have used \Cref{eqn:FieldCreationAnnihilationCommutator}. 

\bbox 
    Another lovely exercise for you: Obtain the first line of \Cref{eqn:KGHamiltonian}. \textit{Hint: Again if you get really stuck, Prof. Tong has done this in his notes, also on page 24.}
\ebox  

\br 
    All jokes aside about the above exercises, it is actually a really beneficial exercise to do and will test your understanding of a reasonable amount of the information leading up to here. So honestly at least give them a go.
\er 

\subsection{A Quantum Field Theorist's Least Favourite Friend: Infinities}

We can extend \Cref{eqn:QHOGroundState} to the field theory case to define the ground state as
\be 
\label{eqn:KGGroundState}
    a_{\vec{p}}\ket{0} = 0 \qquad \forall \vec{p}.
\ee 
We can then use \Cref{eqn:KGHamiltonian} to find the ground state energy: 
\bse 
    H\ket{0} = \bigg[\int d^3\vec{p} \, \frac{1}{2} \omega_{\vec{p}} \del^{(3)}(0)\bigg] \ket{0},
\ese 
which is... infinite?! Uh oh this does not seem good at all, but what did we do wrong? The answer is actually nothing. QFT is filled with infinities and, as Prof. Tong explains "each tells us something important, usually that we're doing something wrong, or asking the wrong question". What we need to do when infinities arise is ask where they come from and what that implies. So let's investigate this one. 

Firstly the bad news... it's actually two infinities! The first one comes from the fact that we're considering the entirety of $\R^3$. So instead let's put our theory inside a box of size $L$. We adopt the usual idea where we prescribe periodic boundary conditions to the sides, and so get the \textit{flat torus}.\footnote{If you're confused why I say flat torus, either google it or think about what happens geometrically when you make these periodic boundary conditions.} Finally we just let $L\to\infty$ to get our result back. We can then use 
\bse 
    (2\pi)^3 \del^{(3)}(0) = \lim_{L\to\infty} \int_{-L/2}^{L/2} d^3\vec{x} \, e^{i\vec{x}\cdot\vec{p}} \bigg|_{\vec{p}=0} = \lim_{L\to\infty} \int_{-L/2}^{L/2}d^3\vec{x} = V,
\ese 
where $V$ is the volume of our theory. Ok so the $(2\pi)^3\del(0)$ infinity is just because we're considering the total energy $E_0$ instead of the \textit{energy density}
\bse 
    \varepsilon_0 := \frac{E_0}{V} = \int \frac{d^3\vec{p}}{(2\pi)^3} \frac{1}{2}\omega_{\vec{p}}.
\ese 
This is still infinite though! Why? well because we take the positive root in \Cref{eqn:KGHarmonicFrequency} and so we're summing an infinite number of positive numbers! This seems like a more tricky beast to tame, however then a light bulb goes off in our heads and we realise that this is just the ground state energy, and so, from the extension of \Cref{eqn:QHOEn}, it will appear in \textit{all} the energies. Then remembering that all we can measure physically is energy \textit{differences} we realise that we are free to just `take this away' from every energy in our system (as the difference wont be effected), so this is what we shall do. 

You might not be very comfortable with `subtracting infinity', and if that is the case, allow me to provide a calming remark: we are mere mortals pretending to be Gods. In less poetic (and perhaps offensive) words: we have assumed that the theory we have written down is valid up to arbitrarily large momentum. This corresponds to arbitrarily large energies, or arbitrarily low length scales. Nature is highly unlikely to agree with us and so plays the trump card of "you need to cut-off your theory at high momenta!" If we do this, the ground state energy density will become finite and then we can comfortably take it away from all energies and proceed as if nothing ever happened. Infinities of this kind are called \textit{ultra-violet divergences}. 

\subsection{Finally, The Spectrum}

Now that we have tamed our infinities, we can proceed to finding the spectrum of the Klein-Gordan field. Continuing with the extension of the QHO, we have 
\be 
\label{eqn:KGHamiltonianAAdaggerCommutators}
    [H,a_{\vec{p}}\,]\ket{0} = -\omega_{\vec{p}}\, a_{\vec{p}}\ket{0}, \qand [H,a^{\dagger}_{\vec{p}}\,] = \omega_{\vec{p}}\, a^{\dagger}_{\vec{p}}\ket{0},
\ee 
which encourages us to define 
\bse 
    \ket{\vec{p}} := a^{\dagger}_{\vec{p}}\, \ket{0}.
\ese 
But now we have different creation/annihilation operators and so can excite the ground state in different ways. We therefore use the notation 
\bse 
    \ket{\vec{p}, \vec{q}, ...} := \Big(a^{\dagger}_{\vec{p}} \, a^{\dagger}_{\vec{q}} \,  ... \Big)\ket{0}.
\ese 
Then finally using \Cref{eqn:KGHamiltonianAAdaggerCommutators} gives us 
\bse 
    H\ket{\vec{p}, \vec{q},...} = (\omega_{\vec{p}} + \omega_{\vec{q}} + ... )\ket{0}.
\ese 
This exhausts our spectrum.\footnote{And perhaps it has exhausted you getting to this point.}

\br 
    Just as with the states $\ket{n}$ defined in \Cref{eqn:QHOExcitedStates}, our states $\ket{\vec{p}}$ are not normalised. They are orthogonal though. We shall return to the normalisation later.
\er 

\subsection{Interpreting The Eigenstates: Particles}

So we have our spectrum, now we want to interpret what they mean physically. Well let's focus on the first excited state
\bse 
    \ket{\vec{p}} = a^{\dagger}_{\vec{p}}\ket{0}.
\ese 
We have already seen that this has energy 
\bse 
    E_{\vec{p}} = \omega_{\vec{p}} = \sqrt{\vec{p}^2 +m^2},
\ese 
and have already remarked that this is the energy of a relativistic particle. But could this just be some coincidence? Well let's look at the \textit{physical} momentum given by \Cref{eqn:EnergyAndMomentumCharges}. Straight forward calculation using the Lagrangian for our system gives 
\bse 
    T^{\mu\nu} = \p^{\mu}\phi\p^{\nu}\phi - \eta^{\mu\nu}\cL,
\ese
and so the momentum is 
\bse 
    P^i = \int d^3\vec{x} \, \dot{\phi}(x) \p^i \phi(x),
\ese 
which as a operator we can write\footnote{Recall that we're using the field theorist's convention of $(+,-,-,-)$.}
\bse 
    \vec{P} = - \int d^3\vec{x} \, \pi(\vec{x})\nabla\phi(\vec{x})
\ese
If we then use \Cref{eqn:phipicreationannihilation}, we get 
\bse 
    \begin{split}
        \vec{P} & = \int \frac{d^3\vec{p}}{(2\pi)^3} \frac{\vec{p}}{2}\Big[ a^{\dagger}_{\vec{p}} \, a_{\vec{p}} +  a_{\vec{p}} \, a^{\dagger}_{\vec{p}}  + a_{\vec{p}}\, a_{\vec{-p}} + a^{\dagger}_{\vec{p}}\, a^{\dagger}_{\vec{-p}} \Big] \\
        & = \int \frac{d^3\vec{p}}{(2\pi)^3}\vec{p} a^{\dagger}_{\vec{p}} \, a_{\vec{p}},
    \end{split}
\ese 
where to get to the last line we have used the commutation relation between $a/a^{\dagger}$ and then dropped the $\del(0)$ term as we did for the energy, and then `dropped' the $aa$ and $a^{\dagger}a^{\dagger}$ terms using the argument of \Cref{rem:Sandwiching}. So we see that the first excited state has momentum 
\bse 
    \vec{P}\ket{\vec{p}} = \vec{p}\ket{\vec{p}},
\ese 
which is exactly what we expect for a particle. We therefore \textit{interpret} the state $\ket{\vec{p}}$ to be a particle with 3-momentum $\vec{p}$. It is important to note that this is simply an interpretation, what we're really dealing with is an excitation of a field. 

\bbox 
    Using the definition 
    \bse 
        J^i = \epsilon^{ijk} \int d^3\vec{x} \, (\cJ^0)^{jk},
    \ese 
    where $\epsilon^{ijk}$ is the Levi-Civita tensor density and $\cJ$ is the classical angular momentum defined in \Cref{eqn:LorentzCurrents}, show that the state $\ket{\vec{p}=0}$ has 
    \bse 
        J^i\ket{\vec{p}=0} = 0.
    \ese 
\ebox 

The result of the above exercise tells us that the our interpreted particle has no spin (i.e. no internal angular momentum). That is, our quantisation of the Klein-Gordan field gives rise to a spin-0 particle. 

\subsection{Multiparticle States}

Above we just considered the first excited state, but we have already seen that we can apply the creation operators multiple times, i.e. we have 
\bse 
    \ket{\vec{p}_1,...,\vec{p}_n} = a^{\dagger}_{\vec{p}_1}...a^{\dagger}_{\vec{p}_n}\ket{0}.
\ese 
As we have used $n$ creation operators, we refer to these states are $n$-particle states and interpret it as $n$ particles with 3-momenta $(\vec{p}_1,...,\vec{p}_n)$. As the creation operators commute with each other we have 
\bse 
    \ket{\vec{p},\vec{q}} = \ket{\vec{q},\vec{p}},
\ese 
and so the particles are symmetric under interchange. We therefore see that these particles are \textit{bosons}. This also agrees with the fact that they are spin-0, as per the above exercise. 

Now recall that in the first lecture we showed when we unite special relativity with QM we have to account for particle number changing. We said that we do this by changing our Hilbert space to be a Fock space, \Cref{eqn:FockSpace}. We can clarify a bit what we meant there now. If $\cH$ is the Hilbert space of our 1-particle states (i.e. $\ket{\vec{p}}\in\cH$), then we have
\bse 
    \ket{\vec{p}_1,...,\vec{p}_n} \in \cH^{\otimes n}.
\ese 
We then take the direct sum over all these different states so that our total Hilbert space accounts for the particle creation process. That is, let's say we start off with a state $\ket{\psi}\in\cH^{\otimes n}$ and then something happens to it and causes the production of a new particle. The new state would then be an element of $\cH^{\otimes (n+1)}$. If we don't take the Fock space construction as above, this process would mean that we leave our Hilbert space, which is a big no-no.

\subsection{Interpretation Of $\phi(\vec{x})\ket{0}$}

The next thing we can ask is how does the action of $\phi(\vec{x})$ fit into our particle interpretation? Well let's calculated it and see. We have 
\bse 
    \begin{split}
        \phi(\vec{x})\ket{0} & = \int \frac{d^3\vec{p}}{(2\pi)^3} \frac{1}{\sqrt{2\omega_{\vec{p}}}} \Big[ a_{\vec{p}} \, e^{i\vec{p}\cdot\vec{x}} + a^{\dagger}_{\vec{p}} \, e^{-i\vec{p}\cdot\vec{x}} \Big]\ket{0} \\
        & = \int \frac{d^3\vec{p}}{(2\pi)^3} \frac{1}{\sqrt{2\omega_{\vec{p}}}} e^{-i\vec{p}\cdot\vec{x}} \ket{\vec{p}}.
    \end{split}
\ese 
Now compare this to the result from QM\footnote{This result comes from inserting a complete set of states, $\b1 = \int dp \ket{p}\bra{p}$. See any decent QM book for details.}
\bse 
    \ket{x} = \int dp e^{-ipx} \ket{p}.
\ese 
We can therefore interpret $\phi(\vec{x})\ket{0}$ as a particle at position $\vec{x}$. 

\subsection{Number Operator}

We have just talked about constructing a whole Fock space to take into account the fact that particle number can change. Well in our free theory this construction is actually not important because nothing is interacting and so there's no chance for more particles to be produced. We can show this explicitly by defining the \textit{number operator}
\be 
\label{eqn:NumberOperator}
    N := \int \frac{d^3p}{(2\pi)^3} a^{\dagger}_{\vec{p}} \, a_{\vec{p}},
\ee 
which satisfies 
\bse 
    N\ket{\vec{p}_1,...,\vec{p}_n} = n\ket{\vec{p}_1,...,\vec{p}_n}.
\ese 
The proof that particle number is conserved in a free theory is the content of the next exercise. 

\bbox 
    Using the above definition and \Cref{eqn:KGHamiltonian} (with the delta term dropped) show that 
    \be
    \label{eqn:NumberOperatorConserved}
        [H,N] = 0.
    \ee 
\ebox  

This tells us that once we are in a particular $\cH^{\otimes n}$ in the Fock space, we will never leave it in a free theory. This makes them a bit boring to study. Fortunately later we will allow for interactions in our theory and see that the above result no longer holds in general. The sad news is free theories are the only ones we can solve exactly, and so in order to study interacting theories we are going to have to develop the mathematics of Feynman diagrams. These are essentially life saving tools to calculate things in perturbation expansions of interacting QFTs. 
\input{sections/IFT4.restoringlorentz.tex}
\input{sections/IFT5.propagators.tex}
\chapter{Interaction Picture \& Dyson's Formula}

So far we have used the Schr\"{o}dinger picture and the Heisenberg picture. Recall that the difference between these two pictures was where the time dependence sat: for the Schr\"{o}dinger picture the states are time dependant, whereas in the Heisenberg picture it is the operators that are time dependant. There is a third picture, the \textit{interaction picture},  which is a sort of hybrid of the other two. As the name suggests, it is useful when studying interacting systems. What we do is consider small perturbations from some well-understood Hamiltonian, $H_0$. That is, we write the Hamiltonian as
\bse 
    H = H_0 + H_{\text{int}}.
\ese 
We treat $H_{\text{int}}$ like a Schr\"{o}dinger Hamiltonian, and $H_0$ like a Heisenberg one. That is, the time dependence of the states is governed by $H_{\text{int}}$ while the time dependence of the operators is governed by $H_0$. As we did for the other pictures, we denote the states/operators in the interaction picture with a subscript $I$, and they are given as follows:
\be 
\label{eqn:InteractionPictureStatesAndOperators}
    \ket{\psi(t)}_I = e^{iH_0t}\ket{\psi(t)}_S, \qand \cO_I(t) = e^{iH_0t}\cO_Se^{-iH_0t}.
\ee 
From this we ca work out how the states $\ket{\psi}_I$ depend on time. We simply use the Schr\"{o}dinger equation
\bse 
    i\frac{d}{dt}\ket{\psi}_S = H_S\ket{\psi}_s
\ese 
along with $H_S=(H_0+H_{\text{int}})_S$. We get 
\bse 
    \begin{split}
        i\frac{d}{dt} \Big( e^{-iH_0t}\ket{\psi}_I\Big) & = \big(H_0+H_{\text{int}}\big)_S e^{-iH_0t}\ket{\psi}_I \\
        H_0\Big(e^{-iH_0t}\ket{\psi}_I\Big) + e^{-iH_0t}\bigg(i\frac{d}{dt}\ket{\psi}_I\bigg) & = \big(H_0+H_{\text{int}}\big)_S e^{-iH_0t}\ket{\psi}_I,
    \end{split}
\ese
so we get
\be
\label{eqn:InteractionSchrodigerEquation}
    i\frac{d}{dt}\ket{\psi}_I = (H_{\text{int}})_I\ket{\psi}_I, \qquad \text{with} \qquad (H_{\text{int}})_I := e^{iH_ot} (H_{\text{int}})_S e^{-iH_0t}.
\ee 

\bnn 
    From now on we shall simply write $H_I(t)$ instead of $(H_{\text{int}})_I(t)$ for obvious reasons. However we should be careful not to confuse $H_I(t)$ with $H_{\text{int}}$ itself: the former is the latter in the interaction picture, while the latter is a small perturbation to our Hamiltonian. Note in particular that the former is \textit{always} a function of time whereas the latter is time independent in the Schr\"{o}dinger picture. 
\enn 

\bbox 
    Convince yourself that $[H_I(t_1),H_I(t_2)]\neq0$ in general. 
\ebox 

Ok this looks great but it's useless unless we can actually solve it, so what do we do? The answer is we derive what is known as \textit{Dyson's formula}.

\section{Dyson's Formula}

We start with the ansatz\footnote{As with all good physics derivations do, our guess will turn out to miraculously be correct...}
\bse 
    \ket{\psi(t)}_I = U(t,t_0) \ket{\psi(t_0)}_I,
\ese 
where $U(t,t_0)$ is a \textit{unitary time evolution operator} satisfying 
\bse 
    U(t_1,t_2)U(t_2,t_3) = U(t_1,t_3), \qand U(t,t) = \b1.
\ese 
Using \Cref{eqn:InteractionSchrodigerEquation}, we can easily show 
\be 
\label{eqn:USchrodingerEquation}
    i\frac{d}{dt}U(t,t_0) = H_I(t)U(t,t_0).
\ee 
Now if we were just considering wavefunctions here instead of operators then the solution to this differential equation would just be 
\be
\label{eqn:UWrong}
    U(t,t_0) = \exp\bigg( -i\int_{t_0}^t H_I(t') dt'\bigg).
\ee
However we are considering non-commuting operators, and so we have to be careful about ordering. To be more explicit, consider the Taylor expansion of the above formula, we get 
\bse 
    U(t,t_0) = \b1 - i\int_{t_0}^t H_I(t')dt' + \frac{(-i)^2}{2}\bigg(\int_{t_0}^t H_I(t')dt'\bigg)^2 + ...,
\ese 
and we want the derivative to leave us with 
\bse 
    i\frac{d}{dt}U(t,t_0) = H_I(t) \Bigg[ \b1 - i\int_{t_0}^t H_I(t')dt' + \frac{(-i)^2}{2}\bigg(\int_{t_0}^t H_I(t')dt'\bigg)^2 + ...\Bigg].
\ese 
However if we take the derivative, we get terms like
\bse 
    -\frac{1}{2}\bigg(\int_{t_0}^t H_I(t')dt'\bigg)H_I(t) - \frac{1}{2}H_I(t) \bigg(\int_{t_0}^t H_I(t')dt'\bigg).
\ese 
The second term is the kind of thing we want but the first term has the $H_I(t)$ on the wrong side! We cannot simply `pull it through' as the operators are non-commuting. So we need to alter \Cref{eqn:UWrong} somehow to account for this. The obvious\footnote{Or perhaps only obvious when you know its the answer...} thing to try is taking a time ordering. 

\bcl 
    The solution of \Cref{eqn:USchrodingerEquation} is given by \textit{Dyson's formula}
    \be 
    \label{eqn:DysonsFormula}
        U(t,t_0) = \cT \exp\bigg( -i\int_{t_0}^t H_I(t') dt'\bigg),
    \ee 
    where $\cT$ is the time ordering operator defined above, \Cref{eqn:TimeOrdering}.
\ecl 

\br 
    Before presenting the proof, let's make a quick remark. Recall that if the result of an integral is finite, then we can use\footnote{If you haven't seen this before, it's worth convincing yourself why this is true.}
    \bse 
        \bigg(\int_a^b f(x) dx\bigg)^2 = \int_a^bdx \int_a^b dy f(x)f(y).
    \ese 
    Now in QFT we consider bounded operators, and so the integral over the Hamiltonian is finite, so we can turn our exponential expansion above into a sum of higher order integrals, rather then powers of integrals. This is obviously a \textit{massive} help. In particular, our expansion becomes
    \bse 
        U(t,t_0) = \b1 - i\int_{t_0}^t dt'H_I(t') + \frac{(-i)^2}{2}\Bigg[ \int_{t_0}^t dt' \int_{t'}^t dt'' H_I(t'')H_I(t') + \int_{t_0}^t dt' \int_{t_0}^{t'} dt'' H_I(t')H_I(t'')\Bigg] + ...
    \ese 
    Note the integration limits on the quadratic terms. The first one is the condition that $t''>t'$ as the lower limit for $t''$ is $t'$, while the second one is the condition that $t'>t''$ as the upper limit for $t''$ is $t'$. These two terms are in fact equal as
    \bse
        \begin{split}
            \int_{t_0}^t dt' \int_{t'}^t dt'' H_I(t'')H_I(t') & = \int_{t_0}^t dt'' \int_{t_0}^{t''} dt' H_I(t'')H_I(t') \\
            & = \int_{t_0}^t dt' \int_{t_0}^{t'} dt'' H_I(t')H_I(t''),
        \end{split}
    \ese 
    where the left had side says $t''$ has lower limit $t'$ while the first expression on the right hand side says $t'$ has supper limit $t''$. These are clearly the same statement. The last expression is obtained by simple change of variables $t' \longleftrightarrow t''$. We can also see this diagrammatically as the two following triangle areas: the blue area corresponds to the left hand side above (i.e. $t''>t')$, whereas the red area is the final expression (i.e. $t'>t''$).
    \begin{center}
        \btik 
            \draw[thick, ->] (-0.5,0) -- (4,0);
            \node at (4,-0.3) {$t'$};
            \draw[thick, ->] (0,-0.5) -- (0,4);
            \node at (-0.3,4) {$t''$};
            \draw[fill = blue, opacity = 0.5] (1,1) -- (3,3) -- (1,3) -- (1,1);
            \draw[fill = red, opacity = 0.5] (1,1) -- (3,3) -- (3,1) -- (1,1);
            \draw[thick] (1,1) -- (3,1) -- (3,3) -- (1,3) -- (1,1);
            \draw (1,1) -- (3,3);
            \draw[dashed] (0,1) -- (1,1) -- (1,0);
            \node at (-0.3,1) {$t_0$};
            \node at (1,-0.3) {$t_0$};
            \draw[dashed] (0,3) -- (1,3);
            \node at (-0.3,3) {$t$};
            \draw[dashed] (3,0) -- (3,1);
            \node at (3,-0.3) {$t$};
        \etik 
    \end{center}
    So our expansion becomes 
    \bse 
        U(t,t_0) = \b1 -i \int_{t_0}^t dt'H_I(t') + (-i)^2 \int_{t_0}^tdt'\int_{t_0}^{t'} dt'' H_I(t')H_I(t'') + ...
    \ese 
\er 

\bq 
    Given the final result from the above remark, the proof of Dyson's formula is almost trivial. We simply note that our left most integral is \textit{always} the one with upper limit $t$, and so when we act with the time derivative we are always just getting the latest time result (as the expression is time ordered). We can therefore pull it out of the time ordering and get the result. In terms of maths, that is 
    \bse 
        \begin{split}
            i\frac{d}{dt}\Bigg[ \cT\exp \bigg(-i\int_{t_0}^t H_I(t')dt'\bigg)\Bigg] & = \cT\bigg[H_I(t) \exp \bigg(-i\int_{t_0}^t H_I(t')dt'\bigg)\Bigg] \\
            & = H_I(t)\Bigg[\cT\exp \bigg(-i\int_{t_0}^t H_I(t')dt'\bigg)\Bigg],
        \end{split}
    \ese 
    which is the result we want. 
\eq 

Dyson's formula might look a bit daunting, but when we remember that in the interaction picture we're considering small perturbations, we can truncate our expansion and it becomes a lot nicer. Now, recalling that we said in order to study interacting filed theories we were going to take a perturbation around a free theory, we see why Dyson's formula is so useful. It will allow us to use something called \textit{Wick's theorem} and then define the all-so-useful Feynman diagrams. But first let's talk about scattering. 

\section{Scattering}

\bd
    We define the \textit{scattering matrix}, or \textit{S-matrix} for short, to be the amplitude to go from some initial state $\ket{i}$ to a final state $\ket{f}$:
    \be 
    \label{eqn:SMatrixI}
        S_{fi} := \bra{f}S\ket{i} := \lim_{t_{\pm}\to\infty}\bra{f(t_+)} U(t_+,t_-) \ket{i(t_-)}_I,
    \ee 
    where the states are in the interaction picture
\ed 

When we study scattering in QFT we make one big assumption:
\mybox{
    \begin{center}
        The initial and final states, which we collectively call \textit{asymptotic states}, are eigenstates of the free Hamiltonian. .
    \end{center}
}
\noindent The basic idea is we want to say that the initially the fields/particles are so far apart that they do not interact with each other at all, and so do not feel in the interaction Hamiltonian. As Prof. Tong points out,\footnote{Pages 54-55.} at first this seems like a reasonable thing to do, but then we realise it's actually not as solid an assumption as we might think. We shall illustrate one of the problems here. 

Consider the case of a real Klein-Gordan field, $\phi$, in some potential $V(x)$, the Lagrangian is 
\bse 
    \cL = \cL_{KG} - V(x)\phi,
\ese 
then the equations of motion become 
\bse 
    (\p^2+m^2)\phi = -V(x).
\ese 
So our asymptotic states condition tells us that we need $V(x)\to 0$ as $|x|\to\infty$. This still seems somewhat reasonable. However now let's consider the interaction term to be a $\phi^4$ term. Recall the Lagrangian is 
\bse 
    \cL = \cL_{KG} - \frac{\l}{4!}\phi^4,
\ese
and you should have shown in the exercise that the equation of motion is
\bse 
    (\p^2+m^2)\phi = -\frac{\l}{3!}\phi^3.
\ese 
This poses a more significant problem, as we can't `turn off' our interaction at $|t|\to\infty$; the field is defined everywhere! Despite this we continue on with our assumption that the asymptotic states are eigenstates of the free theory. We therefore rewrite \Cref{eqn:SMatrix} as
\mybox{
    \be 
    \label{eqn:SMatrix}
        S_{fi} := \bra{f}S\ket{i} := \lim_{t_{\pm}\to\infty}\bra{f(t_+)} U(t_+,t_-) \ket{i(t_-)},
    \ee 
}
\noindent where the states are now eigenstates for the free theory.

We can actually define a different matrix, called the \textit{transition matrix}, by 
\bse 
    T_{fi} := \bra{f}(S-\b1)\ket{i},
\ese 
which removes the $\b1$ term in the Taylor expansion of $U(t,t_0)$.

\subsection{Particle Decay}

Ok let's actually derive our first interaction result. Consider scalar Yakawa theory, and recall that the Lagrangian is 
\bse 
    \cL = \frac{1}{2}(\p\phi)^2 -m^2\phi^2 + \p_{\mu}\psi^*\p^{\mu}\psi - M\psi^*\psi - g\psi^*\psi\phi.
\ese 
This has\footnote{This notation is meant to mean everything in the bracket is a function of $x$. Just want to save writing all the brackets and clutterig notation.} 
\bse 
    H_{\text{int}} = g \int d^3x \, \psi^{\dagger}_x\psi_x\phi_x.
\ese 
We're going to `flip' the diagram we gave for this at the end of last lecture, and consider the decay process 
\bse 
    \phi(p) \to \overline{\psi}(q_2)\psi(q_1),
\ese 
where $\phi$ is the particle created by $a^{\dagger}$, $\psi$ the one created by $b^{\dagger}$ and $\overline{\psi}$ the \textit{anti}particle created by $c^{\dagger}$. The diagram\footnote{We will understand how to draw these soon.} looks like 
\begin{center}
    \btik 
        \midarrow (0,0) -- (1,1);
        \draw[->] (0.4,0.2) -- (0.9,0.7);
        \node at (0.8,0.3) {$q_1$};
        \node at (1.2,1.3) {$\psi$};
        \midarrow (1,-1) -- (0,0);
        \draw[->] (0.4,-0.2) -- (0.9,-0.7);
        \node at (1.2,-1.2) {$\overline{\psi}$};
        \node at (0.8,-0.25) {$q_2$};
        \draw[thick,dashed] (-1.5,0) -- (0,0);
        \node at (-1.7,0) {$\phi$};
        \draw[->] (-1,0.2) -- (-0.4,0.2);
        \node at (-0.7,0.5) {$p$};
        \draw[fill=black] (0,0) circle [radius=0.07cm];
        \node at (-0.15,-0.3) {$g$};
    \etik 
\end{center}
Our asymptotic states are given by 
\bse 
    \begin{split}
        \ket{i} & = \sqrt{2E_{\vec{p}}} \, a_{\vec{p}}^{\dagger} \, \ket{0} =: \ket{\phi(p)} \\
        \ket{i} & = \sqrt{4E_{\vec{q}_1}E_{\vec{q}_2}} \,  b_{\vec{q}_1}^{\dagger} \, c_{\vec{q}_2}^{\dagger} \, \ket{0} =: \ket{\psi(q_1)\overline{\psi}(q_2)}
    \end{split}
\ese 
So what is our scattering amplitude? Well we are just considering the first order expansion, i.e. one power of $H_I$ (higher powers would correspond to terms like $\overline{\psi}(p_1)\psi(p_2) \to \phi(q) \to \overline{\psi}(p_3)\psi(p_4)$ etc), so we have
\bse 
    U(-\infty,\infty) = \b1 - ig\int d^4 x \psi^{\dagger}_x\psi_x\phi_x + \cO(g^2),
\ese
and so the transition matrix to leading order in $g$ is 
\bse 
    T_{fi} = ig \bra{f} \int d^4 x \, \psi^{\dagger}_x \psi_x \phi_x \ket{i}. 
\ese 
We now need to substitute in $\psi_x^{\dagger}$, $\psi_x$ and $\phi_x$. First let's just do $\phi_x$. We use the definition 
\bse 
    \phi(x) = \int \frac{d^3\vec{k}}{(2\pi)^3} \frac{1}{\sqrt{2E_{\vec{k}}}} \Big( a_{\vec{k}} \, e^{-ikx} + a^{\dagger}_{\vec{k}} \, e^{ikx}\Big),
\ese 
and note that the $a_{\vec{k}}^{\dagger}$ term will act on $\ket{i}$ and produce a two particle state. That is (up to factors of $\sqrt{2E}$)
\bse 
    a^{\dagger}_{\vec{k}} \ket{i} = a^{\dagger}_{\vec{k}} \, a^{\dagger}_{\vec{p}} \ket{0} = \ket{\phi(p)\phi(k)}.
\ese
Then noting that the $\psi/\psi^{\dagger}$ terms are not going to be able to `undo' this, we will get a zero projection onto the final state $\ket{f}$. So we just need to consider the $a_{\vec{k}}$ term. So using the usual trick of commuting $a_{\vec{k}}$ with the $a_{\vec{p}}^{\dagger}$ and picking up a $(2\pi)^3\del^{(3)}(\vec{k}-\vec{p})$, we get 
\bse 
    T_{fi} = -ig \bra{f} \int d^4x \, \psi^{\dagger}_x\psi_x e^{-ipx}\ket{0}.
\ese 

\bbox 
    Using the definitions for $\psi$ and $\psi^{\dagger}$ show that 
    \bse 
        \bra{f}\psi^{\dagger}_x\psi_x \, ``=" \, e^{i(q_1+q_2)x}\bra{0}.
    \ese 
    \textit{Hint: Again you will argue away terms that will vanish when taking the inner product. This is why I have put ``=''.}
\ebox 

Using the result of the above exercise we get 
\bse 
    T_{fi} = -ig \int d^4 x \, e^{i(q_1+q_2-p)x} = -ig(2\pi)^4\del^{(4)}(p-q_1-q_2).
\ese 
This result is basically just telling us that 4-momentum must be conserved at the vertex, a very nice physical result. What kind of restrictions does this put on the process? Well the result is Lorentz invariant so we can pick the decaying $\phi$s rest frame to evaluate it. In this frame we have $p=(m,0,0,0)$, so the delta function splits into 
\bse 
    \del^{(4)}(p-q_1-q_2) = \del(m-q_1^0-q_2^0)\del^{(3)}(\vec{q}_1+\vec{q}_2),
\ese 
and so we see the decay can only happen if $m \geq 2M$.

Obviously we expect momentum conservation to be something that comes up a lot in our interaction calculations, and indeed it turns out that in general
\be 
\label{eqn:TransitionMatrixMatrixElements}
    T_{fi} = i (2\pi)^4 M_{fi} \, \del^{(4)}(p_i - p_f),
\ee 
where $p_i$ and $p_f$ are the total 4-momentum initial and final state, respectively. So essentially our goal is to find the $M_{fi}$s, and these are given the, somewhat disappointing, name \textit{matrix elements}. We often also group the $i$ with it and call $iM_{fi}$ the matrix elements.

\subsection{$\phi^4$ Scattering}

Ok so we've computed our first scattering matrix, but it corresponded to a decay, so now let's find the first \textit{scattering} scattering matrix. That is we want something like 
\bse 
    F(p_1)G(p_2) \to F(p_3)G(p_4),
\ese 
where $F$ and $G$ are some fields. We could consider higher order terms in the expansion for the scalar Yakawa theory as mentioned above, however as we will see these are incredibly hard to compute and will require the machinery of Wick contractions. So we want something that to first order in $H_I$ will give us 4 fields (two in, two out). The easiest example is of course $\phi^4$ theory, which has 
\bse 
    H_{\text{int}} = \frac{\l}{4!} \int d^3 x \, \phi_x\phi_x\phi_x\phi_x,
\ese 
and corresponds simply to the scattering process 
\bse 
    \phi(p_1)\phi(p_2) \to \phi(p_3)\phi(p_4).
\ese 
The calculation of the transition matrix is left as an exercise. 

\bbox 
    Show that the above interaction Hamiltonian leads to 
    \bse 
        T_{fi} = -i\l \del^{(4)}(p_1+p_2-p_3-p_4).
    \ese 
\ebox  

Note that the factor of $4!$ has gone in the result to the above exercise. This is actually the reason it's included in the original expression, so that the matrix elements don't have any additional factors. Its an example of a what we call \textit{symmetry factors}, which will become more clear after studying Wick contractions. 

\br 
    Note that for the scalar Yakawa decay we could have actually used the S-matrix instead of the transition matrix. This is because the two only differ by the inner product between the initial and final state, which for the scalar Yakawa theory vanishes. However for the $\phi^4$ scattering, this inner product is non-vanishing and doesn't actually correspond to a scattering. Diagrammatically it looks like,
    \begin{center}
        \btik 
            \draw[thick] (-2,0.75) -- (2,0.75);
            \node at (-2.3,0.75) {$\phi$};
            \node at (2.3,0.75) {$\phi$};
            \draw[->] (-1,1) -- (1,1) node [midway, above] {$p_1=p_3$};
            \draw[thick] (-2,-0.75) -- (2,-0.75);
            \node at (-2.3,-0.75) {$\phi$};
            \node at (2.3,-0.75) {$\phi$};
            \draw[->] (-1,-1) -- (1,-1) node [midway, below] {$p_2=p_4$};
        \etik 
    \end{center}
    It is therefore important to actually distinguish between $S_{fi}$ and $T_{fi}$, however it is often the case that we write $S_{fi}$ and really mean $T_{fi}$. In these notes I shall try be careful here, but its quite possible that I will forget and make a mistake, so keep on your toes. 
\er 


\input{sections/IFT7.wick.tex}
\chapter{LSZ Theorem, Loop \& Amputated Diagrams}

Last lecture we discussed fully connected diagrams and derived the Feynman rules for $\psi\psi \to \psi\psi$ scattering. We also said that there were two other types of conceptually different diagrams, namely \textit{connected} and \textit{disconnected} diagrams. As diagrams they look like:
\begin{center}
    \btik 
        \begin{scope}[xshift=-3.25cm]
            \draw (-3,1.5) -- (3,1.5) -- (3,-3.5) -- (-3,-3.5) -- (-3,1.5);
            \midarrow (-2,0) -- (-1,0);
            \node at (-2.2,0) {$\psi$};
            \midarrow (-1,0) -- (1,0);
            \draw[thick, dashed] (-1,0) .. controls (-0.5,1) and (0.5,1) .. (1,0);
            \node at (0,1) {$\phi$};
            \midarrow (1,0) -- (2,0);
            \node at (0,-0.3) {$\psi$};
            \node at (2.2,0) {$\psi$};
            \midarrow (-2,-2) -- (2,-2);
            \node at (-2.2,-2) {$\psi$};
            \node at (2.2,-2) {$\psi$};
            \node at (0,-3) {\large{Connected}};
        \end{scope}
        %
        \begin{scope}[xshift=3.25cm]
            \draw (-2,1.5) -- (6,1.5) -- (6,-3.5) -- (-2,-3.5) -- (-2,1.5);
            \midarrow (-1,0) -- (1,0);
            \node at (-1.2,0) {$\psi$};
            \node at (1.2,0) {$\psi$};
            \midarrow (-1,-2) -- (1,-2);
            \node at (-1.2,-2) {$\psi$};
            \node at (1.2,-2) {$\psi$};
            \beforemidarrow (3,-1) circle [radius=0.5cm];
            \node at (3,-0.25) {$\psi$};
            \draw[thick, dashed] (3.5,-1) -- (4.5,-1);
            \node at (4,-1.3) {$\phi$};
            \beforemidarrow (5,-1) circle [radius=0.5cm];
            \node at (5,-0.25) {$\psi$};
            \node at (2,-3) {\large{Disconnected}};
        \end{scope}
    \etik 
\end{center}

Intuitively, we would say these don't really correspond to scattering as the initial states don't interact with each other. However we need some way to actually prove this, and this is where the so-called LSZ theorem\footnote{Named after Harry Lehmann, Kurt Symanzik and Wolfhart Zimmermann.} comes in. First we need some definitions and a proposition.

\bd[$n$-Point Green's Function]
    We define the \textit{$n$-point Green's function} to be the time ordered vacuum expectation value of $n$ Heisenberg picture\footnote{Note we need this, otherwise time-ordering doesn't mean anything.} field operator. That is 
    \be  
    \label{eqn:nPointGreensFunction}
        G_n(x_1,...,x_n) := \bra{0} \cT\big[\phi(x_1) ... \phi(x_n)\big] \ket{0}.
    \ee 
\ed 

\bd[Wavefunction Normalisation]
    We define the so-called \textit{wavefunction normalisation} to be 
    \bse 
        Z := |\bra{p_i} \phi(x) \ket{0}|^2,
    \ese 
    so that 
    \be 
    \label{eqn:ZDefinition}
        \bra{p_1,...,p_n} \phi(t=\pm\infty,\Vec{x}) \ket{q_1, ..., q_m} = \lim_{t\to\pm\infty} Z^{-1/2} \, \bra{p_1,...,p_n} \phi(x) \ket{q_1, ..., q_m}.
    \ee 
\ed 

\br 
    What \Cref{eqn:ZDefinition} does is it allows us to turn an asymptotic field $\phi(t=\pm\infty,\Vec{x})$ into a field at a general time under a limit. The factor of $\sqrt{Z}$ is included just to normalise the result, hence the name. \textcolor{red}{For Michael: Is this the correct interpretation of $Z$?}
\er 

\bd[Left-Right Derivative Action]\footnote{This might not be a technical name, it's just what I've decided to call it.}
    Let $\phi_1(x)$ and $\phi_2(x)$ be to fields on the spacetime. Then we define the \textit{left-right derivative action} as 
    \be 
    \label{eqn:LeftRightDerivativeAction}
        \phi_1(x) \lra{\p}_{\mu} \, \phi_2(x) := \phi_1(x) \p_{\mu} \phi_2(x) - \big(\p_{\mu}\phi_1(x)\big)\phi_2(x).
    \ee 
\ed 

\bp 
    For the real scalar field, \Cref{eqn:HeisenbergPhi}, we have 
    \be 
    \label{eqn:aaDaggerLeftRightDerivativeAction}
        \begin{split}
            a_{\Vec{p}} & = \frac{i}{\sqrt{2E_{\vec{p}}}} \, \int d^3 \Vec{x} \, e^{ip\cdot x} \lra{\p}_0 \, \phi(x) \\
            a_{\Vec{p}}^{\dagger} & = -\frac{i}{\sqrt{2E_{\vec{p}}}} \, \int d^3 \Vec{x} \, e^{-ip\cdot x} \lra{\p}_0 \, \phi(x)
        \end{split}
    \ee 
\ep 

\bbox 
    Prove \Cref{eqn:aaDaggerLeftRightDerivativeAction}. \textit{Hint: You will want to use the constraint $p_0=E_{\vec{p}}$ present in \Cref{eqn:HeisenbergPhi} along with $E_{\vec{p}} = \sqrt{\vec{p}^2+m^2} = E_{-\vec{p}}$ at the end.}
\ebox 

We choose here to first state the result of the LSZ theorem and then derive it, as this way the heavy manipulations that follow have some guiding light. 

\bt[LSZ Reduction Formula]
    We can express the S-matrix between an initial $m$-particle state and a $n$-particle final particle state in terms of the $(n+m)$-point Green's function as follows:
    \be 
    \label{eqn:LSZ}
        \begin{split}
            S_{fi} = & \big(iZ^{-1/2}\big)^{n+m} \int dy_1...dy_n \int dx_1...dx_m \, e^{i(p_j\cdot y_j - q_i\cdot x_i)} \\
            & \times \bigg(\prod_{i=1}^m (\p_{x_i}^2+m^2)\bigg) \bigg(\prod_{j=1}^n (\p_{y_j} +m^2) \bigg) G_{n+m}(y_1,...,y_n,x_1,...,x_m) \\
            & + \text{all non-fully connected stuff},
        \end{split}
    \ee 
    where we assume an implicit sum in the exponential.
\et 

Before proving this theorem it is worth stating what it allows us to do. Basically we note that the right-hand side of \Cref{eqn:LSZ} splits into a contribution that contains only information about the fully connected diagrams (we will prove this), and a part that contains only the non-fully connected stuff. We can therefore simply restrict our analysis to processes that are only fully connected in a well defined way. That is, a priori there is no clear way to see that the result of the full S-matrix calculation wont `mix' fully connected and non-fully connected stuff, but the LSZ theorem tells us that it indeed does. This statement does \textit{not} mean that the connected diagrams do not contribute to the full S-matrix value, but simply that we are allowed to restrict ourselves to considering only fully-connected diagrams.

\bq 
    Unfortunately for those who don't like long proofs, this one comes from sheer brute force,\footnote{I'm following the one given by Prof. Weigand, starting on page 49.} so prepare yourself. Firstly we need to clarify the notation that follows: by Dysons formula, the S-matrix is given by a bunch of time evolution operators, and we also take our states to asymptotic (i.e. at $t\to \pm\infty$). Therefore the creation operators that we extract from our initial states \textit{do not} simply act on the final state and annihilate them, as they are separated in time. For this reason we shall use subscripts `in' and `out' on the creation/annihilation operators to avoid confusion. Similarly we will denote the initial/final states using the following notation 
    \bse 
        \bra{p_1...p_n;out}, \qand \ket{q_1...q_m;in}.
    \ese 
    We use this notation as it is the one used by Prof. Weigand in his notes. 
    
    We therefore have 
    \bse 
        \begin{split}
            S_{fi} & = \braket{p_1...p_n;out}{q_1...q_m;in} \\
            & = \sqrt{2E_{\vec{q}_1}}\bra{p_1...p_n;out} a^{\dagger}_{\vec{q}_1} \ket{q_2...q_m;in} \\
            & = \frac{1}{i} \int d^3\vec{x}_1 \bra{p_1...p_n;out} e^{-iq_1\cdot x_1} \lra{\p}_0 \, \phi(t=-\infty,\vec{x}) \ket{q_2...q_m;in} \\
            & = \frac{1}{i} \lim_{t\to-\infty} Z^{-1/2} \, \int d^3\vec{x}_1 \bra{p_1...p_n;out} e^{-iq_1\cdot x_1} \lra{\p}_0 \, \phi(x) \ket{q_2...q_m;in} \\
            & = \frac{1}{i\sqrt{Z}} \lim_{t\to-\infty}  \, \int d^3\vec{x}_1 e^{-iq_1\cdot x_1} \lra{\p}_0  \bra{p_1...p_n;out} \phi(x) \ket{q_2...q_m;in}
        \end{split}
    \ese 
    where we have used the definition $\ket{q} = \sqrt{2E_{\vec{q}}}a^{\dagger}_{\vec{q}}\ket{0}$ and \Cref{eqn:aaDaggerLeftRightDerivativeAction,eqn:ZDefinition}. Note on the third line we have $\phi(t=-\infty,\vec{x})$, this is because we want to consider an initial state, which we take to be at $t=-\infty$. Indeed for this exact reason, we should really denote the creation operator on the second line as something like 
    \bse 
        (a_{\text{in}})^{\dagger}_{\vec{q}_1},
    \ese 
    so that we know it only acts on in/initial states.
    
    What we now want to do is get the $\phi(x)$ to act on a final state, in which case we need to change the limit from $t\to-\infty$ to $t\to + \infty$. We achieve this by using the simple result 
    \bse 
        \lim_{t\to-\infty} f(t) = \lim_{t\to+\infty} f(t) - \lim_{T\to\infty} \int_{-T}^{T} \frac{d}{dt}f(t),
    \ese 
    so our S-matrix splits into two terms:
    \bse 
        \begin{split}
            S_{fi} = & \frac{1}{i\sqrt{Z}} \lim_{t\to+\infty} \, \int d^3\vec{x}_1 e^{-iq_1\cdot x_1} \lra{\p}_0 \bra{p_1...p_n;out} \phi(x) \ket{q_2...q_m;in} \\
            & \qquad + \frac{1}{i\sqrt{Z}} \, \int d^4x_1 \, \p_0\Big( e^{-iq_1\cdot x_1} \lra{\p}_0 \bra{p_1...p_n;out} \phi(x) \ket{q_2...q_m;in}\Big),
        \end{split}
    \ese
    where we notice that the second term contains a $4$-integral. Let's call the first $A$ and the second $B$, and consider them in turn. In $A$ we can use \Cref{eqn:ZDefinition} backwards to get 
    \bse 
        A = \frac{1}{i\sqrt{Z}} \int d^3\vec{x}_1 e^{-iq_1\cdot x_1} \lra{\p}_0  \bra{p_1...p_n;out} \phi(t=+\infty,\vec{x}) \ket{q_2...q_m;in},
    \ese 
    and then we can use \Cref{eqn:aaDaggerLeftRightDerivativeAction} backwards to give 
    \bse 
        A = \sqrt{2E_{\vec{q}_1}}  \bra{p_1...p_n;out} (a_{\text{out}})^{\dagger}_{\vec{q}_1} \ket{q_2...q_m;in},
    \ese 
    where, in agreement with the comment made above, we have included a subscript `out' to tell us this acts on the final states. Now the action of a creation operator to the left gives a delta function, and so we get 
    \bse 
        A = 2E_{\vec{q}_1}(2\pi)^3 \sum_{k=1}^{n}\del^{(3)}(\vec{p}_k -\vec{q}_1) \braket{p_1...\hat{p}_k...p_n;out}{q_2...q_m;in},
    \ese 
    where we have used the standard maths notation where a hatted entry in a string of terms is missing (i.e. $\hat{p}_k$ is missing from the final states). This corresponds to a connected, \textit{but not fully connected}, diagram as we have the $k$-th final particle just corresponding to the $1^{\text{st}}$ initial particle (it is just a straight line in a diagram). 
    
    Now what about $B$? Well we use \Cref{eqn:LeftRightDerivativeAction} to get
    \bse
        \begin{split}
            B & = \frac{1}{i\sqrt{Z}}\, \int d^4x_1 \, \p_0\Big( e^{-iq_1\cdot x_1} \p_0 \la...\ra - \p_0\big(e^{-iq_1\cdot x_1}\big)\la...\ra \Big) \\
            & = \frac{1}{i\sqrt{Z}} \, \int d^4x_1 \, \Big( e^{-iq_1\cdot x_1} \p_0^2 \la...\ra - \p_0^2\big(e^{-iq_1\cdot x_1}\big)\la...\ra \Big)
        \end{split}
    \ese 
    Next note that 
    \bse 
        \p_0^2 e^{-ip\cdot x} = -p_0^2 e^{-ip\cdot x} = (p^2 +(\vec{p})^2) e^{-ip\cdot x} = (m^2 - \nabla^2)e^{-ip\cdot x},
    \ese 
    which we can apply to the second term on the second line above. We then integrate by parts twice\footnote{Note, as always, we assume boundary terms vanish at spatial infinity.} to move the $\nabla^2$ to act on $\la...\ra$ and then use 
    \bse 
        \p^2 := \p_0^2 - \nabla^2
    \ese 
    to give us 
    \bse 
        B = \frac{1}{i\sqrt{Z}} \int d^4 x_1 e^{-iq_1\cdot x_1} \big(\p^2+m^2)\bra{p_1...p_n;out} \phi(x) \ket{q_2...q_m;in}
    \ese 
    This is exactly of the form of one of the product terms in \Cref{eqn:LSZ}! We can then apply a similar argument to all the other initial states, noting that 
    \bse 
        [a^{\dagger}_{\vec{q}_i},a^{\dagger}_{\vec{q}_j}] = 0 \qquad \forall i,j \in \{1,...,m\},
    \ese 
    to obtain the full product from $1$ to $m$ in \Cref{eqn:LSZ}. 
    
    We now need to do the same thing for the final states. However now we have 
    \bse 
        \bra{p} = \sqrt{2E_{\vec{p}}} \bra{0} (a_{\text{out}})_{\vec{p}},
    \ese
    and from 
    \bse 
        [a_{\vec{p}_i},a^{\dagger}_{\vec{q}_j}] \neq 0
    \ese
    we cannot simply commute the annihilation operators past $\phi(x)$ to get them to act on the initial state, that is 
    \bse 
        \begin{split}
            \bra{p_2...p_n;out}\phi(y_1^0=\infty,&\vec{y}_1)\phi(x_1^0=-\infty,\vec{x}_1)\ket{q_1...q_m;in} \\
            & \neq \bra{p_2...p_n;out}\phi(x_1^0=-\infty,\vec{x}_1)\phi(y_1^0=\infty,\vec{y}_1)\ket{q_1...q_m;in}.
        \end{split}
    \ese 
    This is when we notice that \Cref{eqn:LSZ} contains a Green's function, which contains time-ordering and so we realise we are saved! More explicitly, we will get something of the form 
    \bse 
        \begin{split}
            \lim_{x_1^0\to-\infty} \lim_{y_1^0\to+\infty} \int d^3 \vec{x}_1 \int d^3 \vec{y}_1 e^{-iq_1\cdot x_1} \lra{\p}_{x_1^0} \, e^{ip_1\cdot y_1} \lra{\p}_{y_1^0} \, \bra{p_2...p_n;out}\phi(y_1)\phi(x_1)\ket{q_2...q_m;in} \\
            = \lim_{x_1^0\to-\infty} \int d^3 \vec{x}_1 e^{-iq_1\cdot x_1} \lra{\p}_{x_1^0} \, \lim_{y_1^0\to+\infty} \int d^3 \vec{y}_1  e^{ip_1\cdot y_1} \lra{\p}_{y_1^0} \, \bra{p_2...p_n;out}\phi(y_1)\phi(x_1)\ket{q_2...q_m;in}
        \end{split}
    \ese 
    where we notice that the $x_1^0$ limit is taken after the $y_1^0$ one, so during the latter we can take $x_1^0$ to be some fixed finite value, so we can just consider the $y_1$ part of the expression. Now consider 
    \bse 
        \begin{split}
            \lim_{T\to\infty} \int_{-T}^T dy_1^0 \p_{y_1^0} & \bigg(\int d^3\vec{y}_1 e^{ip_1\cdot y_1} \lra{\p}_{y_1^0} \bra{p_2...p_n;out}\cT\big[ \phi(y_1)\phi(x_1)\big] \ket{q_2...q_m;in}\bigg) \\
            = & \lim_{y_1^0\to+\infty} \int d^3\vec{y}_1 e^{ip_1\cdot y_1} \lra{\p}_{y_1^0} \bra{p_2...p_n;out} \phi(y_1)\phi(x_1) \ket{q_2...q_m;in} \\
            & \qquad  - \lim_{y_1^0\to-\infty} \int d^3\vec{y}_1 e^{ip_1\cdot y_1} \lra{\p}_{y_1^0} \bra{p_2...p_n;out} \phi(x_1)\phi(y_1) \ket{q_2...q_m;in}.
        \end{split}
    \ese 
    The first term on the right hand side is what we want, (compare it to the proof above), while the second term on the right hand side will again give a connected, but not fully connected, diagram as we get a $a_{\vec{p}_1}$ term acting on the initial state. So we finally see that we can replace the expression we had above for the Green's function (the time-ordered expression) at the cost of picking up a connected, but not fully connected, term. Doing this for all the final states will then give us all the terms in the product from $j=1$ to $j=n$ needed in \Cref{eqn:LSZ}. Note also that all the other factors needed for \Cref{eqn:LSZ} also appear, i.e. all the exponentials and the $iZ^{-1/2}$ factors.
\eq 

Ok so we have that the fully connected part of the scattering process is given by a $(n+m)$-point Green's function. However the LSZ formula is quite a large formula and so we want to simplify it somehow. We do this by introducing the \textit{truncated Green's function} as follows
\be 
\label{eqn:TruncatedGreensFunction}
    \begin{split}
        G_{n+m}(y_1,...,y_n,x_1,...,x_m) = \int & d^4 z_1 ... d^4 dz_{m+n} \Delta_F(x_1-z_1) ... \Delta_F(x_m-z_m) \\
        & \times \Delta(y_1-z_{m+1}) ... \Delta_F(y_n-z_{n+m}) \widetilde{G}_{n+m}(z_1,...,z_{n+m}),
    \end{split}
\ee
where $\Delta_F$ is the Feynman propagator and $\widetilde{G}_{n+m}$ is our truncated $(n+m)$-point Green's functions. Essentially what we're doing is saying is propagate our initial states to the points $\{z_1,...,z_m\}$ and similarly the final states with $\{z_{m+1},...,z_{m+n}\}$, and characterise all the interaction stuff in the middle using some Green's function, $\widetilde{G}$, as we've tried to indicate below diagrammatically. 
\begin{center}
    \btik[scale=0.8]
        \draw[thick, fill = gray!40, opacity = 0.8] (0,0) circle [radius=2cm];
        \node at (0,0) {\Large{$\widetilde{G}_{5+4}$}};
        \node at (-6.3,0) {$x_3$};
        \node at (-1.7,0) {$z_3$};
        \draw[thick] (-6,0) -- (-2,0);
        \draw[fill=black] (-2,0) circle [radius=0.07cm];
        \begin{scope}[rotate around={20:(0,0)}]
            \node at (-6.3,0) {$x_4$};
            \node at (-1.7,0) {$z_4$};
            \draw[thick] (-6,0) -- (-2,0);
            \draw[fill=black] (-2,0) circle [radius=0.07cm];
        \end{scope}
        \begin{scope}[rotate around={40:(0,0)}]
            \node at (-6.3,0) {$x_5$};
            \node at (-1.7,0) {$z_5$};
            \draw[thick] (-6,0) -- (-2,0);
            \draw[fill=black] (-2,0) circle [radius=0.07cm];
        \end{scope}
        \begin{scope}[rotate around={-20:(0,0)}]
            \node at (-6.3,0) {$x_2$};
            \node at (-1.7,0) {$z_2$};
            \draw[thick] (-6,0) -- (-2,0);
            \draw[fill=black] (-2,0) circle [radius=0.07cm];
        \end{scope}
        \begin{scope}[rotate around={-40:(0,0)}]
            \node at (-6.3,0) {$x_1$};
            \node at (-1.7,0) {$z_1$};
            \draw[thick] (-6,0) -- (-2,0);
            \draw[fill=black] (-2,0) circle [radius=0.07cm];
        \end{scope}
        %
        \node at (6.3,0) {$y_3$};
        \node at (1.7,0) {$z_8$};
        \draw[thick] (6,0) -- (2,0);
        \draw[fill=black] (2,0) circle [radius=0.07cm];
        \begin{scope}[rotate around={20:(0,0)}]
            \node at (6.3,0) {$y_2$};
            \node at (1.7,0) {$z_7$};
            \draw[thick] (6,0) -- (2,0);
            \draw[fill=black] (2,0) circle [radius=0.07cm];
        \end{scope}
        \begin{scope}[rotate around={55:(0,0)}]
            \node at (6.3,0) {$y_1$};
            \node at (1.7,0) {$z_6$};
            \draw[thick] (6,0) -- (2,0);
            \draw[fill=black] (2,0) circle [radius=0.07cm];
        \end{scope}
        \begin{scope}[rotate around={-30:(0,0)}]
            \node at (6.3,0) {$y_4$};
            \node at (1.7,0) {$z_9$};
            \draw[thick] (6,0) -- (2,0);
            \draw[fill=black] (2,0) circle [radius=0.07cm];
        \end{scope}
    \etik 
\end{center}

Why would we introduce the truncated Green's function? Well we recall that the Feynman propagator is a Green's function of the Klein-Gordan operator:
\bse 
    (\p^2+m^2)\Delta_F(x-z) = 0,
\ese 
so in the LSZ formula all of the Klein-Gordan operators will just give us delta functions, which we can integrate over, leaving us with 
\be
\label{eqn:SMatrixTruncatedGreensFunction}
    \begin{split}
        S_{if}^{F.C.} & = \big(iZ^{-1/2}\big)^{n+m} \int dy_1...dy_n\int dx_1...dx_m e^{i(p_j\cdot y_j - q_i\cdot x_i)} \widetilde{G}_{n+m}(x_1...,x_m, y_1,...,y_n) \\
        & = \big(iZ^{-1/2}\big)^{n+m} \widetilde{G}_{n+m} (q_1,...,q_m, p_1, ... , p_n),
    \end{split}
\ee 
where the second line follows from the delta functions we get from the exponentials, and where the superscript F.C. means "fully connected". This result tells us that we can compute the entire fully connected S-matrix by just considering the momentum space truncated Green's function! This is a massively convenient result. 

\br 
    It is worth stressing again that the LSZ theorem does \textit{not} tell us that the connected, but not fully connected, diagrams vanish but that we can ignore them if we want to just consider a scattering process where everything interacts. In fact the former diagrams will contribute to renormalisation and so are very important. 
\er 

\section{Loop Diagrams}

So far we have only considered what are known as \textit{tree level diagrams}. This essentially just means that all of our vertices are connected to our external lines. However this is clearly not the only type of diagram we can have and we can have vertices internally. For example in the scalar Yakawa theory the following is a valid diagram to order $g^2$:
\begin{center}
    \btik 
        \midarrow (-3,0) -- (0,0);
        \node at (-3.3,0) {$\psi$};
        \draw[->] (-2,0.3) -- (-1,0.3) node [midway, above] {$p_1$};
        \midarrow (0,0) -- (3,0);
        \node at (3.3,0) {$\psi$};
        \draw[->] (1,0.3) -- (2,0.3) node [midway, above] {$p_3$};
        \draw[thick, dashed] (0,0) -- (0,-1);
        \draw[->] (-0.3,-0.15) -- (-0.3,-0.85) node [midway, left] {$k_1$};
        \midarrow (0,-1.5) circle [radius=0.5cm];
        \draw[->] (0.8,-1.5) arc (0:60:0.7cm);
        \node at (1,-1) {$\ell_1$};
        \draw[->, rotate around={180:(0,-1.5)}] (0.8,-1.5) arc (0:60:0.7cm);
        \node at (-1,-2) {$\ell_2$};
        \midarrow (0,-1.5) circle [radius=-0.5cm];
        \draw[thick, dashed] (0,-2) -- (0,-3);
        \draw[->] (0.3,-2.15) -- (0.3,-2.85) node [midway, right] {$k_2$};
        \midarrow (-3,-3) -- (0,-3);
        \node at (-3.3,-3) {$\psi$};
        \draw[->] (-2,-3.3) -- (-1,-3.3) node [midway, below] {$p_2$};
        \midarrow (0,-3) -- (3,-3);
        \node at (3.3,-3) {$\psi$};
        \draw[->] (1,-3.3) -- (2,-3.3) node [midway, below] {$p_4$};
    \etik 
\end{center}
\noindent where we haven't labelled the internal particles and the $(-ig)$ factors to avoid cluttering the diagram. This is called a \textit{loop diagram} for obvious reasons. Loop diagrams are a bit of a pain because we don't have enough delta functions to remove all of our integrals. That is, recall that in the momentum space Feynman rules we include a factor of\footnote{Note that before we only defined this for $\phi$ propagators, but here we say we get the same factor for $\psi$ ones but we simply replace the mass with the $\psi$ mass. From now on we shall use $m_{\phi}=m$ and $m_{\psi}=M$.}
\bse 
    \int \frac{d^4k_i}{(2\pi)^2} \frac{i}{k_i^2-m_{\phi/\psi}^2 + i\epsilon}
\ese
for each internal line and a delta function of the momentum flowing in at each vertex. For example, in the above diagram the $k_1$ and $k_2$ momentum are obtained from the delta functions with the external momenta, i.e. $k_1=p_1-p_3$ and $k_2=p_4-p_2$. However we cannot find the values of $\ell_1$ and $\ell_2$, but at best can obtain a relation between them. This is just because $\ell_1$ and $\ell_2$ will both appear in two delta functions, namely 
\bse 
    \del^{(4)}(k_1+\ell_1-\ell_2), \qand \del^{(4)}(\ell_2-k_2-\ell_1),
\ese
and so when we integrate over one of the two $\ell$s we kill both delta functions and we are left with the integral over the other $\ell$ and no delta function to remove it! To be more explicit, if we did the integral over $\ell_2$, say, we would get 
\bse 
    \ell_2 = k_1 + \ell_1, \qand \ell_2 = k_2 + \ell_1,
\ese 
so if we equate the two terms we just get 
\bse 
    k_1 = k_2 \qquad \implies \qquad p_1 + p_2 = p_3+p_4,
\ese
where the implication comes from the momentum conservation on the $k$s. This just tells us the $4$-momentum initially and finally are equivalent, and does not allow us to conclude what $\ell_1$ is. 

\br 
    As we said when we introduced the momentum space Feynman rules, it is common to alter the rules slightly to so just include the factor 
    \bse 
        \frac{i}{k^2-m^2+i\epsilon}
    \ese 
    for each propagator and then impose momentum conservation at each vertex. If we use these rules, then when we have loops we insert the additional rule of "and then integrate over all undetermined momenta", i.e. $\ell_1$ in the above example. 
\er 

We also get loop diagrams in connected, but not fully connected, diagrams, as in the next exercise. 

\bbox 
    Show that this diagram
    \begin{center}
        \btik 
            \midarrow (-3,0) -- (-1,0);
            \node at (-3.2,0) {$\psi$};
            \midarrow (-1,0) -- (1,0);
            \draw[thick, dashed] (-1,0) .. controls (-0.5,1) and (0.5,1) .. (1,0);
            \draw[->, rotate around={-30:(-1.2,0.2)}] (-1.2,0.2) arc (0:-30:-2) node [midway, above] {$k$};
            \draw[->] (-2.8,-0.3) -- (-1.8,-0.3) node [midway,below] {$p_1$};
            \draw[->] (1.8,-0.3) -- (2.8,-0.3) node [midway,below] {$p_2$};
            \draw[->] (-0.5,-0.3) -- (0.5,-0.3) node [midway,below] {$\ell$};
            \midarrow (1,0) -- (3,0);
            \node at (3.2,0) {$\psi$};
        \etik 
    \end{center}
    \noindent corresponds to 
    \bse 
        (2\pi)^4 \del^{(4)}(p_2-p_1)\Bigg[ \int \frac{d^4\ell}{(2\pi)^4} (-ig)^2 \bigg( \frac{i}{(p_1-\ell)^2- m^2 +i\epsilon} \bigg) \bigg( \frac{i}{\ell^2 - M^2 +i\epsilon} \bigg) \Bigg]
    \ese
\ebox  

\br 
    Note that the result of the previous exercise diverges logarithmically in $\ell$. This is easily seen because we essentially have $1/\ell^4$ and we do $d^4\ell$. This kind of divergence is known as a \textit{UV divergence}.
\er 

\section{Amputated Diagrams}

Note in all the Feynman diagrams we have just drawn, we always draw the ingoing and outgoing particles as `pure' lines, by which I mean nothing is attached to them. An example of something we haven't drawn is the following:
\begin{center}
    \btik 
        \midarrow (-2,0) -- (-1,0);
        \midarrow (-1,0) -- (1,0);
        \draw[thick, dashed] (-1,0) .. controls (-0.5,1) and (0.5,1) .. (1,0);
        \midarrow (1,0) -- (3,0);
        \midarrow (3,0) -- (5,0);
        \draw[thick, dashed] (3,0) -- (3,-2);
        \midarrow (-2,-2) -- (3,-2);
        \midarrow (3,-2) -- (5,-2);
    \etik 
\end{center}

The reason we have never considered these is because what we've been drawing are so-called \textit{amputated diagrams}. You get amputated diagrams by saying "can I cut through the diagram and remove something by only cutting through one line?" If the answer is yes, do it. So for the above example we would cut along the dotted red line below:
\begin{center}
    \btik 
        \midarrow (-2,0) -- (-1,0);
        \midarrow (-1,0) -- (1,0);
        \draw[thick, dashed] (-1,0) .. controls (-0.5,1) and (0.5,1) .. (1,0);
        \midarrow (1,0) -- (3,0);
        \midarrow (3,0) -- (5,0);
        \draw[thick, dashed] (3,0) -- (3,-2);
        \midarrow (-2,-2) -- (3,-2);
        \midarrow (3,-2) -- (5,-2);
        \draw[ultra thick, red, dashed] (3,1) -- (1,-1);
    \etik 
\end{center}
\noindent which gives us the scattering diagram we drew before. It is important that you only cut through $1$ line, i.e. we can not amputate the following 
\begin{center}
    \btik 
        \midarrow (-2,0) -- (-1,0);
        \midarrow (-1,0) -- (0,0);
        \midarrow (0,0) -- (1,0);
        \draw[thick, dashed] (-1,0) .. controls (-0.5,1) and (0.5,1) .. (1,0);
        \midarrow (1,0) -- (2,0);
        \draw[thick, dashed] (0,0) -- (0,-2);
        \midarrow (-2,-2) -- (0,-2);
        \midarrow (0,-2) -- (2,-2);
    \etik 
\end{center}
The reason we amputate diagrams is because the LSZ theorem tells us to consider only amputated diagrams. This is seen from the truncated Greens function expression: we consider just propagating our initial and final states in/out without any interactions and then only consider the interactions that happen between all the particles in the internal parts, see the truncated $\widetilde{G}$ picture above. 

% -------------------------------------------------------------------
% Bibliography/Further Readings
% -------------------------------------------------------------------

\chapter*{Useful Texts \& Further Readings}

\section*{Textbooks}
\begin{itemize}
    \item M. E. Peskin and D. V. Schroeder, “\textit{An Introduction to QFT}” (Addison Wesley, 1995).
    \item C. Itzykson and J.-B. Zuber, “\textit{Quantum Field Theory}” (McGraw-Hill,1980).
    \item L. H. Ryder, “\textit{Quantum Field Theory}” (Cambridge University Press, 1996). 
    \item P. Ramond, “\textit{Field Theory, A Modern Primer}” (Benjamin, 1994).
    \item S. Weinberg, “\textit{Quantum Theory of Fields}”, vol I and II (Cambridge University Press, 1996).
    \item A. Zee, “\textit{Quantum Field Theory in a Nutshell}” (Princeton University Press, 2010).
    \item F. Mandl and G. Shaw, “\textit{Quantum Field Theory}” (Wiley, 1984.
    \item L. H. Ryder, “\textit{Quantum Field Theory}” (Cambridge University Press, 1996).
    \item Schwartz, Matthew D. \textit{Quantum field theory and the standard model} (Cambridge University Press, 2014).
\end{itemize}

\section*{Other Similar Courses Available Online}
\begin{itemize}
    \item Prof. David Tong,  "\href{http://www.damtp.cam.ac.uk/user/tong/qft.html}{Lectures on Quantum Field Theory}", Cambridge University. 
    \item Prof. Michael Luke, "\href{https://www.physics.utoronto.ca/~luke/PHY2403F/References_files/lecturenotes.pdf}{PHY2403F Lecture Notes}", University Of Toronto. 
    \item Prof. Timo Weigand "\href{https://www.thphys.uni-heidelberg.de/~weigand/QFT2-14/SkriptQFT2.pdf}{Quantum Field Theory I + II}", Institute for Theoretical Physics, Heidelberg University. 
\end{itemize}


%\bibliographystyle{agsm} 
%\bibliography{mybibliography} 
%\printbibliography[heading=bibintoc]


% -------------------------------------------------------------------
% Appendices
% -------------------------------------------------------------------

%\begin{appendices}
%\input{sections/appendixA.tex}
%\end{appendices}

\end{document}
